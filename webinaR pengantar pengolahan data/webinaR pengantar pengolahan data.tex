\documentclass[11pt]{article}

    \usepackage[breakable]{tcolorbox}
    \usepackage{parskip} % Stop auto-indenting (to mimic markdown behaviour)
    
    \usepackage{iftex}
    \ifPDFTeX
    	\usepackage[T1]{fontenc}
    	\usepackage{mathpazo}
    \else
    	\usepackage{fontspec}
    \fi

    % Basic figure setup, for now with no caption control since it's done
    % automatically by Pandoc (which extracts ![](path) syntax from Markdown).
    \usepackage{graphicx}
    % Maintain compatibility with old templates. Remove in nbconvert 6.0
    \let\Oldincludegraphics\includegraphics
    % Ensure that by default, figures have no caption (until we provide a
    % proper Figure object with a Caption API and a way to capture that
    % in the conversion process - todo).
    \usepackage{caption}
    \DeclareCaptionFormat{nocaption}{}
    \captionsetup{format=nocaption,aboveskip=0pt,belowskip=0pt}

    \usepackage[Export]{adjustbox} % Used to constrain images to a maximum size
    \adjustboxset{max size={0.9\linewidth}{0.9\paperheight}}
    \usepackage{float}
    \floatplacement{figure}{H} % forces figures to be placed at the correct location
    \usepackage{xcolor} % Allow colors to be defined
    \usepackage{enumerate} % Needed for markdown enumerations to work
    \usepackage{geometry} % Used to adjust the document margins
    \usepackage{amsmath} % Equations
    \usepackage{amssymb} % Equations
    \usepackage{textcomp} % defines textquotesingle
    % Hack from http://tex.stackexchange.com/a/47451/13684:
    \AtBeginDocument{%
        \def\PYZsq{\textquotesingle}% Upright quotes in Pygmentized code
    }
    \usepackage{upquote} % Upright quotes for verbatim code
    \usepackage{eurosym} % defines \euro
    \usepackage[mathletters]{ucs} % Extended unicode (utf-8) support
    \usepackage{fancyvrb} % verbatim replacement that allows latex
    \usepackage{grffile} % extends the file name processing of package graphics 
                         % to support a larger range
    \makeatletter % fix for grffile with XeLaTeX
    \def\Gread@@xetex#1{%
      \IfFileExists{"\Gin@base".bb}%
      {\Gread@eps{\Gin@base.bb}}%
      {\Gread@@xetex@aux#1}%
    }
    \makeatother

    % The hyperref package gives us a pdf with properly built
    % internal navigation ('pdf bookmarks' for the table of contents,
    % internal cross-reference links, web links for URLs, etc.)
    \usepackage{hyperref}
    % The default LaTeX title has an obnoxious amount of whitespace. By default,
    % titling removes some of it. It also provides customization options.
    \usepackage{titling}
    \usepackage{longtable} % longtable support required by pandoc >1.10
    \usepackage{booktabs}  % table support for pandoc > 1.12.2
    \usepackage[inline]{enumitem} % IRkernel/repr support (it uses the enumerate* environment)
    \usepackage[normalem]{ulem} % ulem is needed to support strikethroughs (\sout)
                                % normalem makes italics be italics, not underlines
    \usepackage{mathrsfs}
    

    
    % Colors for the hyperref package
    \definecolor{urlcolor}{rgb}{0,.145,.698}
    \definecolor{linkcolor}{rgb}{.71,0.21,0.01}
    \definecolor{citecolor}{rgb}{.12,.54,.11}

    % ANSI colors
    \definecolor{ansi-black}{HTML}{3E424D}
    \definecolor{ansi-black-intense}{HTML}{282C36}
    \definecolor{ansi-red}{HTML}{E75C58}
    \definecolor{ansi-red-intense}{HTML}{B22B31}
    \definecolor{ansi-green}{HTML}{00A250}
    \definecolor{ansi-green-intense}{HTML}{007427}
    \definecolor{ansi-yellow}{HTML}{DDB62B}
    \definecolor{ansi-yellow-intense}{HTML}{B27D12}
    \definecolor{ansi-blue}{HTML}{208FFB}
    \definecolor{ansi-blue-intense}{HTML}{0065CA}
    \definecolor{ansi-magenta}{HTML}{D160C4}
    \definecolor{ansi-magenta-intense}{HTML}{A03196}
    \definecolor{ansi-cyan}{HTML}{60C6C8}
    \definecolor{ansi-cyan-intense}{HTML}{258F8F}
    \definecolor{ansi-white}{HTML}{C5C1B4}
    \definecolor{ansi-white-intense}{HTML}{A1A6B2}
    \definecolor{ansi-default-inverse-fg}{HTML}{FFFFFF}
    \definecolor{ansi-default-inverse-bg}{HTML}{000000}

    % commands and environments needed by pandoc snippets
    % extracted from the output of `pandoc -s`
    \providecommand{\tightlist}{%
      \setlength{\itemsep}{0pt}\setlength{\parskip}{0pt}}
    \DefineVerbatimEnvironment{Highlighting}{Verbatim}{commandchars=\\\{\}}
    % Add ',fontsize=\small' for more characters per line
    \newenvironment{Shaded}{}{}
    \newcommand{\KeywordTok}[1]{\textcolor[rgb]{0.00,0.44,0.13}{\textbf{{#1}}}}
    \newcommand{\DataTypeTok}[1]{\textcolor[rgb]{0.56,0.13,0.00}{{#1}}}
    \newcommand{\DecValTok}[1]{\textcolor[rgb]{0.25,0.63,0.44}{{#1}}}
    \newcommand{\BaseNTok}[1]{\textcolor[rgb]{0.25,0.63,0.44}{{#1}}}
    \newcommand{\FloatTok}[1]{\textcolor[rgb]{0.25,0.63,0.44}{{#1}}}
    \newcommand{\CharTok}[1]{\textcolor[rgb]{0.25,0.44,0.63}{{#1}}}
    \newcommand{\StringTok}[1]{\textcolor[rgb]{0.25,0.44,0.63}{{#1}}}
    \newcommand{\CommentTok}[1]{\textcolor[rgb]{0.38,0.63,0.69}{\textit{{#1}}}}
    \newcommand{\OtherTok}[1]{\textcolor[rgb]{0.00,0.44,0.13}{{#1}}}
    \newcommand{\AlertTok}[1]{\textcolor[rgb]{1.00,0.00,0.00}{\textbf{{#1}}}}
    \newcommand{\FunctionTok}[1]{\textcolor[rgb]{0.02,0.16,0.49}{{#1}}}
    \newcommand{\RegionMarkerTok}[1]{{#1}}
    \newcommand{\ErrorTok}[1]{\textcolor[rgb]{1.00,0.00,0.00}{\textbf{{#1}}}}
    \newcommand{\NormalTok}[1]{{#1}}
    
    % Additional commands for more recent versions of Pandoc
    \newcommand{\ConstantTok}[1]{\textcolor[rgb]{0.53,0.00,0.00}{{#1}}}
    \newcommand{\SpecialCharTok}[1]{\textcolor[rgb]{0.25,0.44,0.63}{{#1}}}
    \newcommand{\VerbatimStringTok}[1]{\textcolor[rgb]{0.25,0.44,0.63}{{#1}}}
    \newcommand{\SpecialStringTok}[1]{\textcolor[rgb]{0.73,0.40,0.53}{{#1}}}
    \newcommand{\ImportTok}[1]{{#1}}
    \newcommand{\DocumentationTok}[1]{\textcolor[rgb]{0.73,0.13,0.13}{\textit{{#1}}}}
    \newcommand{\AnnotationTok}[1]{\textcolor[rgb]{0.38,0.63,0.69}{\textbf{\textit{{#1}}}}}
    \newcommand{\CommentVarTok}[1]{\textcolor[rgb]{0.38,0.63,0.69}{\textbf{\textit{{#1}}}}}
    \newcommand{\VariableTok}[1]{\textcolor[rgb]{0.10,0.09,0.49}{{#1}}}
    \newcommand{\ControlFlowTok}[1]{\textcolor[rgb]{0.00,0.44,0.13}{\textbf{{#1}}}}
    \newcommand{\OperatorTok}[1]{\textcolor[rgb]{0.40,0.40,0.40}{{#1}}}
    \newcommand{\BuiltInTok}[1]{{#1}}
    \newcommand{\ExtensionTok}[1]{{#1}}
    \newcommand{\PreprocessorTok}[1]{\textcolor[rgb]{0.74,0.48,0.00}{{#1}}}
    \newcommand{\AttributeTok}[1]{\textcolor[rgb]{0.49,0.56,0.16}{{#1}}}
    \newcommand{\InformationTok}[1]{\textcolor[rgb]{0.38,0.63,0.69}{\textbf{\textit{{#1}}}}}
    \newcommand{\WarningTok}[1]{\textcolor[rgb]{0.38,0.63,0.69}{\textbf{\textit{{#1}}}}}
    
    
    % Define a nice break command that doesn't care if a line doesn't already
    % exist.
    \def\br{\hspace*{\fill} \\* }
    % Math Jax compatibility definitions
    \def\gt{>}
    \def\lt{<}
    \let\Oldtex\TeX
    \let\Oldlatex\LaTeX
    \renewcommand{\TeX}{\textrm{\Oldtex}}
    \renewcommand{\LaTeX}{\textrm{\Oldlatex}}
    % Document parameters
    % Document title
    \title{webinaR pengantar pengolahan data}
    
    
    
    
    
% Pygments definitions
\makeatletter
\def\PY@reset{\let\PY@it=\relax \let\PY@bf=\relax%
    \let\PY@ul=\relax \let\PY@tc=\relax%
    \let\PY@bc=\relax \let\PY@ff=\relax}
\def\PY@tok#1{\csname PY@tok@#1\endcsname}
\def\PY@toks#1+{\ifx\relax#1\empty\else%
    \PY@tok{#1}\expandafter\PY@toks\fi}
\def\PY@do#1{\PY@bc{\PY@tc{\PY@ul{%
    \PY@it{\PY@bf{\PY@ff{#1}}}}}}}
\def\PY#1#2{\PY@reset\PY@toks#1+\relax+\PY@do{#2}}

\expandafter\def\csname PY@tok@w\endcsname{\def\PY@tc##1{\textcolor[rgb]{0.73,0.73,0.73}{##1}}}
\expandafter\def\csname PY@tok@c\endcsname{\let\PY@it=\textit\def\PY@tc##1{\textcolor[rgb]{0.25,0.50,0.50}{##1}}}
\expandafter\def\csname PY@tok@cp\endcsname{\def\PY@tc##1{\textcolor[rgb]{0.74,0.48,0.00}{##1}}}
\expandafter\def\csname PY@tok@k\endcsname{\let\PY@bf=\textbf\def\PY@tc##1{\textcolor[rgb]{0.00,0.50,0.00}{##1}}}
\expandafter\def\csname PY@tok@kp\endcsname{\def\PY@tc##1{\textcolor[rgb]{0.00,0.50,0.00}{##1}}}
\expandafter\def\csname PY@tok@kt\endcsname{\def\PY@tc##1{\textcolor[rgb]{0.69,0.00,0.25}{##1}}}
\expandafter\def\csname PY@tok@o\endcsname{\def\PY@tc##1{\textcolor[rgb]{0.40,0.40,0.40}{##1}}}
\expandafter\def\csname PY@tok@ow\endcsname{\let\PY@bf=\textbf\def\PY@tc##1{\textcolor[rgb]{0.67,0.13,1.00}{##1}}}
\expandafter\def\csname PY@tok@nb\endcsname{\def\PY@tc##1{\textcolor[rgb]{0.00,0.50,0.00}{##1}}}
\expandafter\def\csname PY@tok@nf\endcsname{\def\PY@tc##1{\textcolor[rgb]{0.00,0.00,1.00}{##1}}}
\expandafter\def\csname PY@tok@nc\endcsname{\let\PY@bf=\textbf\def\PY@tc##1{\textcolor[rgb]{0.00,0.00,1.00}{##1}}}
\expandafter\def\csname PY@tok@nn\endcsname{\let\PY@bf=\textbf\def\PY@tc##1{\textcolor[rgb]{0.00,0.00,1.00}{##1}}}
\expandafter\def\csname PY@tok@ne\endcsname{\let\PY@bf=\textbf\def\PY@tc##1{\textcolor[rgb]{0.82,0.25,0.23}{##1}}}
\expandafter\def\csname PY@tok@nv\endcsname{\def\PY@tc##1{\textcolor[rgb]{0.10,0.09,0.49}{##1}}}
\expandafter\def\csname PY@tok@no\endcsname{\def\PY@tc##1{\textcolor[rgb]{0.53,0.00,0.00}{##1}}}
\expandafter\def\csname PY@tok@nl\endcsname{\def\PY@tc##1{\textcolor[rgb]{0.63,0.63,0.00}{##1}}}
\expandafter\def\csname PY@tok@ni\endcsname{\let\PY@bf=\textbf\def\PY@tc##1{\textcolor[rgb]{0.60,0.60,0.60}{##1}}}
\expandafter\def\csname PY@tok@na\endcsname{\def\PY@tc##1{\textcolor[rgb]{0.49,0.56,0.16}{##1}}}
\expandafter\def\csname PY@tok@nt\endcsname{\let\PY@bf=\textbf\def\PY@tc##1{\textcolor[rgb]{0.00,0.50,0.00}{##1}}}
\expandafter\def\csname PY@tok@nd\endcsname{\def\PY@tc##1{\textcolor[rgb]{0.67,0.13,1.00}{##1}}}
\expandafter\def\csname PY@tok@s\endcsname{\def\PY@tc##1{\textcolor[rgb]{0.73,0.13,0.13}{##1}}}
\expandafter\def\csname PY@tok@sd\endcsname{\let\PY@it=\textit\def\PY@tc##1{\textcolor[rgb]{0.73,0.13,0.13}{##1}}}
\expandafter\def\csname PY@tok@si\endcsname{\let\PY@bf=\textbf\def\PY@tc##1{\textcolor[rgb]{0.73,0.40,0.53}{##1}}}
\expandafter\def\csname PY@tok@se\endcsname{\let\PY@bf=\textbf\def\PY@tc##1{\textcolor[rgb]{0.73,0.40,0.13}{##1}}}
\expandafter\def\csname PY@tok@sr\endcsname{\def\PY@tc##1{\textcolor[rgb]{0.73,0.40,0.53}{##1}}}
\expandafter\def\csname PY@tok@ss\endcsname{\def\PY@tc##1{\textcolor[rgb]{0.10,0.09,0.49}{##1}}}
\expandafter\def\csname PY@tok@sx\endcsname{\def\PY@tc##1{\textcolor[rgb]{0.00,0.50,0.00}{##1}}}
\expandafter\def\csname PY@tok@m\endcsname{\def\PY@tc##1{\textcolor[rgb]{0.40,0.40,0.40}{##1}}}
\expandafter\def\csname PY@tok@gh\endcsname{\let\PY@bf=\textbf\def\PY@tc##1{\textcolor[rgb]{0.00,0.00,0.50}{##1}}}
\expandafter\def\csname PY@tok@gu\endcsname{\let\PY@bf=\textbf\def\PY@tc##1{\textcolor[rgb]{0.50,0.00,0.50}{##1}}}
\expandafter\def\csname PY@tok@gd\endcsname{\def\PY@tc##1{\textcolor[rgb]{0.63,0.00,0.00}{##1}}}
\expandafter\def\csname PY@tok@gi\endcsname{\def\PY@tc##1{\textcolor[rgb]{0.00,0.63,0.00}{##1}}}
\expandafter\def\csname PY@tok@gr\endcsname{\def\PY@tc##1{\textcolor[rgb]{1.00,0.00,0.00}{##1}}}
\expandafter\def\csname PY@tok@ge\endcsname{\let\PY@it=\textit}
\expandafter\def\csname PY@tok@gs\endcsname{\let\PY@bf=\textbf}
\expandafter\def\csname PY@tok@gp\endcsname{\let\PY@bf=\textbf\def\PY@tc##1{\textcolor[rgb]{0.00,0.00,0.50}{##1}}}
\expandafter\def\csname PY@tok@go\endcsname{\def\PY@tc##1{\textcolor[rgb]{0.53,0.53,0.53}{##1}}}
\expandafter\def\csname PY@tok@gt\endcsname{\def\PY@tc##1{\textcolor[rgb]{0.00,0.27,0.87}{##1}}}
\expandafter\def\csname PY@tok@err\endcsname{\def\PY@bc##1{\setlength{\fboxsep}{0pt}\fcolorbox[rgb]{1.00,0.00,0.00}{1,1,1}{\strut ##1}}}
\expandafter\def\csname PY@tok@kc\endcsname{\let\PY@bf=\textbf\def\PY@tc##1{\textcolor[rgb]{0.00,0.50,0.00}{##1}}}
\expandafter\def\csname PY@tok@kd\endcsname{\let\PY@bf=\textbf\def\PY@tc##1{\textcolor[rgb]{0.00,0.50,0.00}{##1}}}
\expandafter\def\csname PY@tok@kn\endcsname{\let\PY@bf=\textbf\def\PY@tc##1{\textcolor[rgb]{0.00,0.50,0.00}{##1}}}
\expandafter\def\csname PY@tok@kr\endcsname{\let\PY@bf=\textbf\def\PY@tc##1{\textcolor[rgb]{0.00,0.50,0.00}{##1}}}
\expandafter\def\csname PY@tok@bp\endcsname{\def\PY@tc##1{\textcolor[rgb]{0.00,0.50,0.00}{##1}}}
\expandafter\def\csname PY@tok@fm\endcsname{\def\PY@tc##1{\textcolor[rgb]{0.00,0.00,1.00}{##1}}}
\expandafter\def\csname PY@tok@vc\endcsname{\def\PY@tc##1{\textcolor[rgb]{0.10,0.09,0.49}{##1}}}
\expandafter\def\csname PY@tok@vg\endcsname{\def\PY@tc##1{\textcolor[rgb]{0.10,0.09,0.49}{##1}}}
\expandafter\def\csname PY@tok@vi\endcsname{\def\PY@tc##1{\textcolor[rgb]{0.10,0.09,0.49}{##1}}}
\expandafter\def\csname PY@tok@vm\endcsname{\def\PY@tc##1{\textcolor[rgb]{0.10,0.09,0.49}{##1}}}
\expandafter\def\csname PY@tok@sa\endcsname{\def\PY@tc##1{\textcolor[rgb]{0.73,0.13,0.13}{##1}}}
\expandafter\def\csname PY@tok@sb\endcsname{\def\PY@tc##1{\textcolor[rgb]{0.73,0.13,0.13}{##1}}}
\expandafter\def\csname PY@tok@sc\endcsname{\def\PY@tc##1{\textcolor[rgb]{0.73,0.13,0.13}{##1}}}
\expandafter\def\csname PY@tok@dl\endcsname{\def\PY@tc##1{\textcolor[rgb]{0.73,0.13,0.13}{##1}}}
\expandafter\def\csname PY@tok@s2\endcsname{\def\PY@tc##1{\textcolor[rgb]{0.73,0.13,0.13}{##1}}}
\expandafter\def\csname PY@tok@sh\endcsname{\def\PY@tc##1{\textcolor[rgb]{0.73,0.13,0.13}{##1}}}
\expandafter\def\csname PY@tok@s1\endcsname{\def\PY@tc##1{\textcolor[rgb]{0.73,0.13,0.13}{##1}}}
\expandafter\def\csname PY@tok@mb\endcsname{\def\PY@tc##1{\textcolor[rgb]{0.40,0.40,0.40}{##1}}}
\expandafter\def\csname PY@tok@mf\endcsname{\def\PY@tc##1{\textcolor[rgb]{0.40,0.40,0.40}{##1}}}
\expandafter\def\csname PY@tok@mh\endcsname{\def\PY@tc##1{\textcolor[rgb]{0.40,0.40,0.40}{##1}}}
\expandafter\def\csname PY@tok@mi\endcsname{\def\PY@tc##1{\textcolor[rgb]{0.40,0.40,0.40}{##1}}}
\expandafter\def\csname PY@tok@il\endcsname{\def\PY@tc##1{\textcolor[rgb]{0.40,0.40,0.40}{##1}}}
\expandafter\def\csname PY@tok@mo\endcsname{\def\PY@tc##1{\textcolor[rgb]{0.40,0.40,0.40}{##1}}}
\expandafter\def\csname PY@tok@ch\endcsname{\let\PY@it=\textit\def\PY@tc##1{\textcolor[rgb]{0.25,0.50,0.50}{##1}}}
\expandafter\def\csname PY@tok@cm\endcsname{\let\PY@it=\textit\def\PY@tc##1{\textcolor[rgb]{0.25,0.50,0.50}{##1}}}
\expandafter\def\csname PY@tok@cpf\endcsname{\let\PY@it=\textit\def\PY@tc##1{\textcolor[rgb]{0.25,0.50,0.50}{##1}}}
\expandafter\def\csname PY@tok@c1\endcsname{\let\PY@it=\textit\def\PY@tc##1{\textcolor[rgb]{0.25,0.50,0.50}{##1}}}
\expandafter\def\csname PY@tok@cs\endcsname{\let\PY@it=\textit\def\PY@tc##1{\textcolor[rgb]{0.25,0.50,0.50}{##1}}}

\def\PYZbs{\char`\\}
\def\PYZus{\char`\_}
\def\PYZob{\char`\{}
\def\PYZcb{\char`\}}
\def\PYZca{\char`\^}
\def\PYZam{\char`\&}
\def\PYZlt{\char`\<}
\def\PYZgt{\char`\>}
\def\PYZsh{\char`\#}
\def\PYZpc{\char`\%}
\def\PYZdl{\char`\$}
\def\PYZhy{\char`\-}
\def\PYZsq{\char`\'}
\def\PYZdq{\char`\"}
\def\PYZti{\char`\~}
% for compatibility with earlier versions
\def\PYZat{@}
\def\PYZlb{[}
\def\PYZrb{]}
\makeatother


    % For linebreaks inside Verbatim environment from package fancyvrb. 
    \makeatletter
        \newbox\Wrappedcontinuationbox 
        \newbox\Wrappedvisiblespacebox 
        \newcommand*\Wrappedvisiblespace {\textcolor{red}{\textvisiblespace}} 
        \newcommand*\Wrappedcontinuationsymbol {\textcolor{red}{\llap{\tiny$\m@th\hookrightarrow$}}} 
        \newcommand*\Wrappedcontinuationindent {3ex } 
        \newcommand*\Wrappedafterbreak {\kern\Wrappedcontinuationindent\copy\Wrappedcontinuationbox} 
        % Take advantage of the already applied Pygments mark-up to insert 
        % potential linebreaks for TeX processing. 
        %        {, <, #, %, $, ' and ": go to next line. 
        %        _, }, ^, &, >, - and ~: stay at end of broken line. 
        % Use of \textquotesingle for straight quote. 
        \newcommand*\Wrappedbreaksatspecials {% 
            \def\PYGZus{\discretionary{\char`\_}{\Wrappedafterbreak}{\char`\_}}% 
            \def\PYGZob{\discretionary{}{\Wrappedafterbreak\char`\{}{\char`\{}}% 
            \def\PYGZcb{\discretionary{\char`\}}{\Wrappedafterbreak}{\char`\}}}% 
            \def\PYGZca{\discretionary{\char`\^}{\Wrappedafterbreak}{\char`\^}}% 
            \def\PYGZam{\discretionary{\char`\&}{\Wrappedafterbreak}{\char`\&}}% 
            \def\PYGZlt{\discretionary{}{\Wrappedafterbreak\char`\<}{\char`\<}}% 
            \def\PYGZgt{\discretionary{\char`\>}{\Wrappedafterbreak}{\char`\>}}% 
            \def\PYGZsh{\discretionary{}{\Wrappedafterbreak\char`\#}{\char`\#}}% 
            \def\PYGZpc{\discretionary{}{\Wrappedafterbreak\char`\%}{\char`\%}}% 
            \def\PYGZdl{\discretionary{}{\Wrappedafterbreak\char`\$}{\char`\$}}% 
            \def\PYGZhy{\discretionary{\char`\-}{\Wrappedafterbreak}{\char`\-}}% 
            \def\PYGZsq{\discretionary{}{\Wrappedafterbreak\textquotesingle}{\textquotesingle}}% 
            \def\PYGZdq{\discretionary{}{\Wrappedafterbreak\char`\"}{\char`\"}}% 
            \def\PYGZti{\discretionary{\char`\~}{\Wrappedafterbreak}{\char`\~}}% 
        } 
        % Some characters . , ; ? ! / are not pygmentized. 
        % This macro makes them "active" and they will insert potential linebreaks 
        \newcommand*\Wrappedbreaksatpunct {% 
            \lccode`\~`\.\lowercase{\def~}{\discretionary{\hbox{\char`\.}}{\Wrappedafterbreak}{\hbox{\char`\.}}}% 
            \lccode`\~`\,\lowercase{\def~}{\discretionary{\hbox{\char`\,}}{\Wrappedafterbreak}{\hbox{\char`\,}}}% 
            \lccode`\~`\;\lowercase{\def~}{\discretionary{\hbox{\char`\;}}{\Wrappedafterbreak}{\hbox{\char`\;}}}% 
            \lccode`\~`\:\lowercase{\def~}{\discretionary{\hbox{\char`\:}}{\Wrappedafterbreak}{\hbox{\char`\:}}}% 
            \lccode`\~`\?\lowercase{\def~}{\discretionary{\hbox{\char`\?}}{\Wrappedafterbreak}{\hbox{\char`\?}}}% 
            \lccode`\~`\!\lowercase{\def~}{\discretionary{\hbox{\char`\!}}{\Wrappedafterbreak}{\hbox{\char`\!}}}% 
            \lccode`\~`\/\lowercase{\def~}{\discretionary{\hbox{\char`\/}}{\Wrappedafterbreak}{\hbox{\char`\/}}}% 
            \catcode`\.\active
            \catcode`\,\active 
            \catcode`\;\active
            \catcode`\:\active
            \catcode`\?\active
            \catcode`\!\active
            \catcode`\/\active 
            \lccode`\~`\~ 	
        }
    \makeatother

    \let\OriginalVerbatim=\Verbatim
    \makeatletter
    \renewcommand{\Verbatim}[1][1]{%
        %\parskip\z@skip
        \sbox\Wrappedcontinuationbox {\Wrappedcontinuationsymbol}%
        \sbox\Wrappedvisiblespacebox {\FV@SetupFont\Wrappedvisiblespace}%
        \def\FancyVerbFormatLine ##1{\hsize\linewidth
            \vtop{\raggedright\hyphenpenalty\z@\exhyphenpenalty\z@
                \doublehyphendemerits\z@\finalhyphendemerits\z@
                \strut ##1\strut}%
        }%
        % If the linebreak is at a space, the latter will be displayed as visible
        % space at end of first line, and a continuation symbol starts next line.
        % Stretch/shrink are however usually zero for typewriter font.
        \def\FV@Space {%
            \nobreak\hskip\z@ plus\fontdimen3\font minus\fontdimen4\font
            \discretionary{\copy\Wrappedvisiblespacebox}{\Wrappedafterbreak}
            {\kern\fontdimen2\font}%
        }%
        
        % Allow breaks at special characters using \PYG... macros.
        \Wrappedbreaksatspecials
        % Breaks at punctuation characters . , ; ? ! and / need catcode=\active 	
        \OriginalVerbatim[#1,codes*=\Wrappedbreaksatpunct]%
    }
    \makeatother

    % Exact colors from NB
    \definecolor{incolor}{HTML}{303F9F}
    \definecolor{outcolor}{HTML}{D84315}
    \definecolor{cellborder}{HTML}{CFCFCF}
    \definecolor{cellbackground}{HTML}{F7F7F7}
    
    % prompt
    \makeatletter
    \newcommand{\boxspacing}{\kern\kvtcb@left@rule\kern\kvtcb@boxsep}
    \makeatother
    \newcommand{\prompt}[4]{
        \ttfamily\llap{{\color{#2}[#3]:\hspace{3pt}#4}}\vspace{-\baselineskip}
    }
    

    
    % Prevent overflowing lines due to hard-to-break entities
    \sloppy 
    % Setup hyperref package
    \hypersetup{
      breaklinks=true,  % so long urls are correctly broken across lines
      colorlinks=true,
      urlcolor=urlcolor,
      linkcolor=linkcolor,
      citecolor=citecolor,
      }
    % Slightly bigger margins than the latex defaults
    
    \geometry{verbose,tmargin=1in,bmargin=1in,lmargin=1in,rmargin=1in}
    
    

\begin{document}
    
    \maketitle
    
    

    
    {©} 2020 \textbar{} Sandy H.S. Herho dan Dasapta E. Irawan

    \hypertarget{apa-itu-r}{%
\section{Apa itu R?}\label{apa-itu-r}}

    R merupakan bahasa pemrograman dan lingkungan perangkat lunak yang
digunakan untuk analisis statistik, pemodelan data, visualiasi grafik,
dan pelaporan data. R bersifat sumber terbuka dan gratis diunduh oleh
siapapun. Webinar ini ditujukan untuk membantu pemula untuk memulai
pengolahan data dengan menggunakan R.

    \hypertarget{instalasi}{%
\section{Instalasi}\label{instalasi}}

    \begin{enumerate}
\def\labelenumi{\arabic{enumi}.}
\tightlist
\item
  Kunjungi
  https://docs.conda.io/projects/conda/en/latest/user-guide/install/
  untuk instalasi Miniconda versi 3.
\item
  Ikuti prosedur instalasi (jangan lupa atur PATH -nya).
\item
  Buku \emph{Command Line Interface} (CLI), jalankan:
\end{enumerate}

\begin{verbatim}
conda create -n enviRonment r-essential r-base
\end{verbatim}

\begin{enumerate}
\def\labelenumi{\arabic{enumi}.}
\setcounter{enumi}{3}
\tightlist
\item
  Aktifkan lingkungan virtual dengan menjalankan perintah:
\end{enumerate}

\begin{verbatim}
conda activate enviRonment
\end{verbatim}

\begin{enumerate}
\def\labelenumi{\arabic{enumi}.}
\setcounter{enumi}{4}
\tightlist
\item
  Lakukan instalasi Jupyter Notebook:
\end{enumerate}

\begin{verbatim}
conda install -c anaconda jupyter
\end{verbatim}

\begin{enumerate}
\def\labelenumi{\arabic{enumi}.}
\setcounter{enumi}{5}
\tightlist
\item
  Lakukan instalasi IRkernel:
\end{enumerate}

\begin{verbatim}
conda install -c r r-irkernel
\end{verbatim}

\begin{enumerate}
\def\labelenumi{\arabic{enumi}.}
\setcounter{enumi}{6}
\tightlist
\item
  Jalankan perintah:
\end{enumerate}

\begin{verbatim}
jupyter notebook
\end{verbatim}

\begin{enumerate}
\def\labelenumi{\arabic{enumi}.}
\setcounter{enumi}{7}
\tightlist
\item
  Jika sudah terbuka di \emph{browser} masing - masing sesi R interaktif
  dapat dimulai.
\item
  Untuk mengakhiri sesi tekan tombol
  \texttt{\textless{}CTRL\textgreater{}\ +\ C} di CLI, dan jalankan
  perintah: \texttt{(bash)\ \ \ \ \ conda\ deactivate}
\end{enumerate}

    \hypertarget{pengantar}{%
\section{PengantaR}\label{pengantar}}

    \hypertarget{tipe---tipe-data}{%
\subsection{Tipe - tipe data}\label{tipe---tipe-data}}

    \begin{enumerate}
\def\labelenumi{\arabic{enumi}.}
\tightlist
\item
  Logical : \texttt{TRUE}, \texttt{FALSE}
\item
  Numeric: \texttt{1}, \texttt{1.2}, \texttt{1239}
\item
  Integer: \texttt{2},\texttt{3},\texttt{5},\texttt{6}
\item
  Complex: \texttt{4\ +\ 3i}
\item
  Character: \texttt{"Halo"},
  \texttt{\textquotesingle{}Halo\textquotesingle{}}
\item
  Raw: Bytes
\end{enumerate}

    \begin{tcolorbox}[breakable, size=fbox, boxrule=1pt, pad at break*=1mm,colback=cellbackground, colframe=cellborder]
\prompt{In}{incolor}{1}{\boxspacing}
\begin{Verbatim}[commandchars=\\\{\}]
\PY{n}{a} \PY{o}{\PYZlt{}\PYZhy{}} \PY{k+kc}{TRUE}
\PY{n}{a} \PY{o}{=} \PY{l+m}{10}
\PY{n}{a} \PY{o}{\PYZlt{}\PYZhy{}} \PY{l+m}{3} \PY{o}{+} \PY{l+m}{4i}
\end{Verbatim}
\end{tcolorbox}

    \begin{tcolorbox}[breakable, size=fbox, boxrule=1pt, pad at break*=1mm,colback=cellbackground, colframe=cellborder]
\prompt{In}{incolor}{2}{\boxspacing}
\begin{Verbatim}[commandchars=\\\{\}]
\PY{n+nf}{print}\PY{p}{(}\PY{n}{a}\PY{p}{)}
\end{Verbatim}
\end{tcolorbox}

    \begin{Verbatim}[commandchars=\\\{\}]
[1] 3+4i
    \end{Verbatim}

    \begin{tcolorbox}[breakable, size=fbox, boxrule=1pt, pad at break*=1mm,colback=cellbackground, colframe=cellborder]
\prompt{In}{incolor}{3}{\boxspacing}
\begin{Verbatim}[commandchars=\\\{\}]
\PY{n+nf}{class}\PY{p}{(}\PY{n}{a}\PY{p}{)} \PY{c+c1}{\PYZsh{} a nerupakan bagian dari kelas ??}
\end{Verbatim}
\end{tcolorbox}

    'complex'

    
    \begin{tcolorbox}[breakable, size=fbox, boxrule=1pt, pad at break*=1mm,colback=cellbackground, colframe=cellborder]
\prompt{In}{incolor}{4}{\boxspacing}
\begin{Verbatim}[commandchars=\\\{\}]
\PY{n}{b} \PY{o}{\PYZlt{}\PYZhy{}} \PY{l+m}{5}
\PY{n+nf}{class}\PY{p}{(}\PY{n}{b}\PY{p}{)}
\end{Verbatim}
\end{tcolorbox}

    'numeric'

    
    \begin{tcolorbox}[breakable, size=fbox, boxrule=1pt, pad at break*=1mm,colback=cellbackground, colframe=cellborder]
\prompt{In}{incolor}{5}{\boxspacing}
\begin{Verbatim}[commandchars=\\\{\}]
\PY{n}{c} \PY{o}{\PYZlt{}\PYZhy{}} \PY{k+kc}{FALSE}
\PY{n+nf}{class}\PY{p}{(}\PY{n}{c}\PY{p}{)}
\end{Verbatim}
\end{tcolorbox}

    'logical'

    
    \begin{tcolorbox}[breakable, size=fbox, boxrule=1pt, pad at break*=1mm,colback=cellbackground, colframe=cellborder]
\prompt{In}{incolor}{6}{\boxspacing}
\begin{Verbatim}[commandchars=\\\{\}]
\PY{n}{d} \PY{o}{\PYZlt{}\PYZhy{}} \PY{l+m}{3L} 
\PY{n+nf}{class}\PY{p}{(}\PY{n}{d}\PY{p}{)}
\end{Verbatim}
\end{tcolorbox}

    'integer'

    
    Secara \emph{default}, seluruh bilangan yang kita tugaskan ke dalam
suatu variabel akan berupa tipe data numerik. Jika kita ingin menjadikan
bilangan tersebut integer, gunakan \texttt{L}.

    \begin{tcolorbox}[breakable, size=fbox, boxrule=1pt, pad at break*=1mm,colback=cellbackground, colframe=cellborder]
\prompt{In}{incolor}{7}{\boxspacing}
\begin{Verbatim}[commandchars=\\\{\}]
\PY{n}{teks} \PY{o}{\PYZlt{}\PYZhy{}} \PY{l+s}{\PYZdq{}}\PY{l+s}{Halo apa kabar?\PYZdq{}}
\end{Verbatim}
\end{tcolorbox}

    \begin{tcolorbox}[breakable, size=fbox, boxrule=1pt, pad at break*=1mm,colback=cellbackground, colframe=cellborder]
\prompt{In}{incolor}{8}{\boxspacing}
\begin{Verbatim}[commandchars=\\\{\}]
\PY{n}{jawaban} \PY{o}{\PYZlt{}\PYZhy{}} \PY{l+s}{\PYZsq{}}\PY{l+s}{Baik baik, Bro!\PYZsq{}}
\end{Verbatim}
\end{tcolorbox}

    \begin{tcolorbox}[breakable, size=fbox, boxrule=1pt, pad at break*=1mm,colback=cellbackground, colframe=cellborder]
\prompt{In}{incolor}{9}{\boxspacing}
\begin{Verbatim}[commandchars=\\\{\}]
\PY{n}{teks}
\end{Verbatim}
\end{tcolorbox}

    'Halo apa kabar?'

    
    \begin{tcolorbox}[breakable, size=fbox, boxrule=1pt, pad at break*=1mm,colback=cellbackground, colframe=cellborder]
\prompt{In}{incolor}{10}{\boxspacing}
\begin{Verbatim}[commandchars=\\\{\}]
\PY{n+nf}{class}\PY{p}{(}\PY{n}{teks}\PY{p}{)}
\end{Verbatim}
\end{tcolorbox}

    'character'

    
    \begin{tcolorbox}[breakable, size=fbox, boxrule=1pt, pad at break*=1mm,colback=cellbackground, colframe=cellborder]
\prompt{In}{incolor}{11}{\boxspacing}
\begin{Verbatim}[commandchars=\\\{\}]
\PY{n+nf}{class}\PY{p}{(}\PY{n}{jawaban}\PY{p}{)}
\end{Verbatim}
\end{tcolorbox}

    'character'

    
    \begin{tcolorbox}[breakable, size=fbox, boxrule=1pt, pad at break*=1mm,colback=cellbackground, colframe=cellborder]
\prompt{In}{incolor}{12}{\boxspacing}
\begin{Verbatim}[commandchars=\\\{\}]
\PY{n}{a} \PY{o}{\PYZlt{}\PYZhy{}} \PY{n+nf}{charToRaw}\PY{p}{(}\PY{l+s}{\PYZsq{}}\PY{l+s}{Mohon maaf lahir batin Mas bro!\PYZsq{}}\PY{p}{)}
\PY{n}{a}
\end{Verbatim}
\end{tcolorbox}

    
    \begin{verbatim}
 [1] 4d 6f 68 6f 6e 20 6d 61 61 66 20 6c 61 68 69 72 20 62 61 74 69 6e 20 4d 61
[26] 73 20 62 72 6f 21
    \end{verbatim}

    
    \texttt{a} merupakan representasi bagaimana tipe data character disimpan
secara internal di dalam memori komputer.

    \begin{tcolorbox}[breakable, size=fbox, boxrule=1pt, pad at break*=1mm,colback=cellbackground, colframe=cellborder]
\prompt{In}{incolor}{13}{\boxspacing}
\begin{Verbatim}[commandchars=\\\{\}]
\PY{n+nf}{class}\PY{p}{(}\PY{n}{a}\PY{p}{)}
\end{Verbatim}
\end{tcolorbox}

    'raw'

    
    \hypertarget{struktur-data}{%
\subsection{Struktur data}\label{struktur-data}}

    Koleksi dari tipe - tipe data.

\begin{itemize}
\tightlist
\item
  Vectors
\item
  Lists
\item
  Matrix
\item
  Arrays
\item
  Data Frames
\item
  Factors
\end{itemize}

    \hypertarget{pengantar-vektor}{%
\subsubsection{Pengantar Vektor}\label{pengantar-vektor}}

    Koleksi dari objek - objek dengan tipe data yg sama.

    \begin{tcolorbox}[breakable, size=fbox, boxrule=1pt, pad at break*=1mm,colback=cellbackground, colframe=cellborder]
\prompt{In}{incolor}{14}{\boxspacing}
\begin{Verbatim}[commandchars=\\\{\}]
\PY{n}{kendaraan} \PY{o}{\PYZlt{}\PYZhy{}} \PY{n+nf}{c}\PY{p}{(}\PY{l+s}{\PYZsq{}}\PY{l+s}{mobil\PYZsq{}}\PY{p}{,} \PY{l+s}{\PYZsq{}}\PY{l+s}{sepeda motor\PYZsq{}}\PY{p}{,} \PY{l+s}{\PYZsq{}}\PY{l+s}{bus\PYZsq{}}\PY{p}{)}
\PY{n}{kendaraan}
\end{Verbatim}
\end{tcolorbox}

    \begin{enumerate*}
\item 'mobil'
\item 'sepeda motor'
\item 'bus'
\end{enumerate*}


    
    \begin{tcolorbox}[breakable, size=fbox, boxrule=1pt, pad at break*=1mm,colback=cellbackground, colframe=cellborder]
\prompt{In}{incolor}{15}{\boxspacing}
\begin{Verbatim}[commandchars=\\\{\}]
\PY{n+nf}{class}\PY{p}{(}\PY{n}{kendaraan}\PY{p}{)}
\end{Verbatim}
\end{tcolorbox}

    'character'

    
    prioritas tipe data:

characters \textgreater{} numeric \textgreater{} integer \textgreater{}
logical

    \begin{tcolorbox}[breakable, size=fbox, boxrule=1pt, pad at break*=1mm,colback=cellbackground, colframe=cellborder]
\prompt{In}{incolor}{16}{\boxspacing}
\begin{Verbatim}[commandchars=\\\{\}]
\PY{n}{kendaraan} \PY{o}{\PYZlt{}\PYZhy{}} \PY{n+nf}{c}\PY{p}{(}\PY{l+m}{3}\PY{p}{,} \PY{l+s}{\PYZsq{}}\PY{l+s}{sepeda motor\PYZsq{}}\PY{p}{,} \PY{l+s}{\PYZsq{}}\PY{l+s}{bus\PYZsq{}}\PY{p}{)}
\PY{n}{kendaraan} \PY{c+c1}{\PYZsh{} 3 dikonversi ke character}
\end{Verbatim}
\end{tcolorbox}

    \begin{enumerate*}
\item '3'
\item 'sepeda motor'
\item 'bus'
\end{enumerate*}


    
    \begin{tcolorbox}[breakable, size=fbox, boxrule=1pt, pad at break*=1mm,colback=cellbackground, colframe=cellborder]
\prompt{In}{incolor}{17}{\boxspacing}
\begin{Verbatim}[commandchars=\\\{\}]
\PY{n}{num} \PY{o}{\PYZlt{}\PYZhy{}} \PY{n+nf}{c}\PY{p}{(}\PY{l+m}{1L}\PY{p}{,} \PY{l+m}{1.5}\PY{p}{,} \PY{l+m}{2}\PY{p}{)}
\PY{n}{num}
\end{Verbatim}
\end{tcolorbox}

    \begin{enumerate*}
\item 1
\item 1.5
\item 2
\end{enumerate*}


    
    \begin{tcolorbox}[breakable, size=fbox, boxrule=1pt, pad at break*=1mm,colback=cellbackground, colframe=cellborder]
\prompt{In}{incolor}{18}{\boxspacing}
\begin{Verbatim}[commandchars=\\\{\}]
\PY{n+nf}{class}\PY{p}{(}\PY{n}{num}\PY{p}{)}
\end{Verbatim}
\end{tcolorbox}

    'numeric'

    
    \hypertarget{mendefinisikan-vektor}{%
\subsubsection{Mendefinisikan vektor}\label{mendefinisikan-vektor}}

    \hypertarget{menggunakan}{%
\paragraph{\texorpdfstring{Menggunakan
\texttt{:}}{Menggunakan :}}\label{menggunakan}}

    Sintaks \(\rightarrow\) \texttt{awal:akhir}

    \begin{tcolorbox}[breakable, size=fbox, boxrule=1pt, pad at break*=1mm,colback=cellbackground, colframe=cellborder]
\prompt{In}{incolor}{19}{\boxspacing}
\begin{Verbatim}[commandchars=\\\{\}]
\PY{n}{x} \PY{o}{\PYZlt{}\PYZhy{}} \PY{l+m}{1}\PY{o}{:}\PY{l+m}{7}
\PY{n}{x}
\end{Verbatim}
\end{tcolorbox}

    \begin{enumerate*}
\item 1
\item 2
\item 3
\item 4
\item 5
\item 6
\item 7
\end{enumerate*}


    
    \begin{tcolorbox}[breakable, size=fbox, boxrule=1pt, pad at break*=1mm,colback=cellbackground, colframe=cellborder]
\prompt{In}{incolor}{20}{\boxspacing}
\begin{Verbatim}[commandchars=\\\{\}]
\PY{n}{x} \PY{o}{\PYZlt{}\PYZhy{}} \PY{l+m}{2} \PY{o}{:} \PY{l+m}{\PYZhy{}2}
\PY{n}{x}
\end{Verbatim}
\end{tcolorbox}

    \begin{enumerate*}
\item 2
\item 1
\item 0
\item -1
\item -2
\end{enumerate*}


    
    \hypertarget{menggunakan-fungsi-seq}{%
\subsubsection{\texorpdfstring{Menggunakan fungsi
\texttt{seq()}}{Menggunakan fungsi seq()}}\label{menggunakan-fungsi-seq}}

    Sintaks \(\rightarrow\) \texttt{seq(from,\ to,\ by,\ length.out)}

\begin{enumerate}
\def\labelenumi{\arabic{enumi}.}
\tightlist
\item
  \texttt{from}: awal
\item
  \texttt{to} : akhir
\item
  \texttt{by} : peningkatan (secara default 1).
\item
  \texttt{length.out}: panjang sekuen.
\end{enumerate}

    \begin{tcolorbox}[breakable, size=fbox, boxrule=1pt, pad at break*=1mm,colback=cellbackground, colframe=cellborder]
\prompt{In}{incolor}{21}{\boxspacing}
\begin{Verbatim}[commandchars=\\\{\}]
\PY{n+nf}{seq}\PY{p}{(}\PY{l+m}{5}\PY{p}{,}\PY{l+m}{10}\PY{p}{)}
\end{Verbatim}
\end{tcolorbox}

    \begin{enumerate*}
\item 5
\item 6
\item 7
\item 8
\item 9
\item 10
\end{enumerate*}


    
    \begin{tcolorbox}[breakable, size=fbox, boxrule=1pt, pad at break*=1mm,colback=cellbackground, colframe=cellborder]
\prompt{In}{incolor}{22}{\boxspacing}
\begin{Verbatim}[commandchars=\\\{\}]
\PY{n+nf}{seq}\PY{p}{(}\PY{n}{from} \PY{o}{=} \PY{l+m}{0}\PY{p}{,} \PY{n}{to} \PY{o}{=} \PY{l+m}{10}\PY{p}{,} \PY{n}{by} \PY{o}{=} \PY{l+m}{2}\PY{p}{)}
\end{Verbatim}
\end{tcolorbox}

    \begin{enumerate*}
\item 0
\item 2
\item 4
\item 6
\item 8
\item 10
\end{enumerate*}


    
    \begin{tcolorbox}[breakable, size=fbox, boxrule=1pt, pad at break*=1mm,colback=cellbackground, colframe=cellborder]
\prompt{In}{incolor}{23}{\boxspacing}
\begin{Verbatim}[commandchars=\\\{\}]
\PY{n+nf}{seq}\PY{p}{(}\PY{n}{from} \PY{o}{=} \PY{l+m}{1}\PY{p}{,} \PY{n}{to} \PY{o}{=} \PY{l+m}{3}\PY{p}{,} \PY{n}{length.out} \PY{o}{=} \PY{l+m}{5}\PY{p}{)}
\PY{c+c1}{\PYZsh{} dibagi jadi 5 bag. dgn jarak yg sama}
\end{Verbatim}
\end{tcolorbox}

    \begin{enumerate*}
\item 1
\item 1.5
\item 2
\item 2.5
\item 3
\end{enumerate*}


    
    \begin{tcolorbox}[breakable, size=fbox, boxrule=1pt, pad at break*=1mm,colback=cellbackground, colframe=cellborder]
\prompt{In}{incolor}{24}{\boxspacing}
\begin{Verbatim}[commandchars=\\\{\}]
\PY{n+nf}{seq}\PY{p}{(}\PY{n}{from}\PY{o}{=}\PY{l+m}{0}\PY{p}{,} \PY{n}{to}\PY{o}{=}\PY{l+m}{20}\PY{p}{,} \PY{n}{length.out} \PY{o}{=} \PY{l+m}{5}\PY{p}{)}
\end{Verbatim}
\end{tcolorbox}

    \begin{enumerate*}
\item 0
\item 5
\item 10
\item 15
\item 20
\end{enumerate*}


    
    Contoh lainnya:

    \begin{tcolorbox}[breakable, size=fbox, boxrule=1pt, pad at break*=1mm,colback=cellbackground, colframe=cellborder]
\prompt{In}{incolor}{25}{\boxspacing}
\begin{Verbatim}[commandchars=\\\{\}]
\PY{n}{x} \PY{o}{\PYZlt{}\PYZhy{}} \PY{l+m}{1}\PY{o}{:}\PY{l+m}{8}
\PY{n}{x}
\end{Verbatim}
\end{tcolorbox}

    \begin{enumerate*}
\item 1
\item 2
\item 3
\item 4
\item 5
\item 6
\item 7
\item 8
\end{enumerate*}


    
    \begin{tcolorbox}[breakable, size=fbox, boxrule=1pt, pad at break*=1mm,colback=cellbackground, colframe=cellborder]
\prompt{In}{incolor}{26}{\boxspacing}
\begin{Verbatim}[commandchars=\\\{\}]
\PY{n}{x} \PY{o}{\PYZlt{}\PYZhy{}} \PY{n+nf}{seq}\PY{p}{(}\PY{l+m}{1}\PY{p}{,}\PY{l+m}{8}\PY{p}{)}
\PY{n}{x}
\end{Verbatim}
\end{tcolorbox}

    \begin{enumerate*}
\item 1
\item 2
\item 3
\item 4
\item 5
\item 6
\item 7
\item 8
\end{enumerate*}


    
    \begin{tcolorbox}[breakable, size=fbox, boxrule=1pt, pad at break*=1mm,colback=cellbackground, colframe=cellborder]
\prompt{In}{incolor}{27}{\boxspacing}
\begin{Verbatim}[commandchars=\\\{\}]
\PY{n}{x} \PY{o}{\PYZlt{}\PYZhy{}} \PY{n+nf}{seq}\PY{p}{(}\PY{l+m}{1}\PY{p}{,}\PY{l+m}{8}\PY{p}{,} \PY{n}{by}\PY{o}{=}\PY{l+m}{2}\PY{p}{)}
\PY{n}{x}
\end{Verbatim}
\end{tcolorbox}

    \begin{enumerate*}
\item 1
\item 3
\item 5
\item 7
\end{enumerate*}


    
    \begin{tcolorbox}[breakable, size=fbox, boxrule=1pt, pad at break*=1mm,colback=cellbackground, colframe=cellborder]
\prompt{In}{incolor}{28}{\boxspacing}
\begin{Verbatim}[commandchars=\\\{\}]
\PY{n}{x} \PY{o}{\PYZlt{}\PYZhy{}} \PY{n+nf}{seq}\PY{p}{(}\PY{l+m}{1}\PY{p}{,}\PY{l+m}{8}\PY{p}{,} \PY{n}{length.out} \PY{o}{=} \PY{l+m}{10}\PY{p}{)}
\PY{n}{x}
\end{Verbatim}
\end{tcolorbox}

    \begin{enumerate*}
\item 1
\item 1.77777777777778
\item 2.55555555555556
\item 3.33333333333333
\item 4.11111111111111
\item 4.88888888888889
\item 5.66666666666667
\item 6.44444444444444
\item 7.22222222222222
\item 8
\end{enumerate*}


    
    \hypertarget{operator}{%
\subsubsection{Operator}\label{operator}}

    \begin{tcolorbox}[breakable, size=fbox, boxrule=1pt, pad at break*=1mm,colback=cellbackground, colframe=cellborder]
\prompt{In}{incolor}{29}{\boxspacing}
\begin{Verbatim}[commandchars=\\\{\}]
\PY{n}{v} \PY{o}{\PYZlt{}\PYZhy{}} \PY{l+m}{1}\PY{o}{:}\PY{l+m}{3}
\PY{n}{w} \PY{o}{\PYZlt{}\PYZhy{}} \PY{l+m}{4}\PY{o}{:}\PY{l+m}{6}
\end{Verbatim}
\end{tcolorbox}

    \begin{tcolorbox}[breakable, size=fbox, boxrule=1pt, pad at break*=1mm,colback=cellbackground, colframe=cellborder]
\prompt{In}{incolor}{30}{\boxspacing}
\begin{Verbatim}[commandchars=\\\{\}]
\PY{n+nf}{print}\PY{p}{(}\PY{n}{v}\PY{p}{)}
\PY{n+nf}{print}\PY{p}{(}\PY{n}{w}\PY{p}{)}
\end{Verbatim}
\end{tcolorbox}

    \begin{Verbatim}[commandchars=\\\{\}]
[1] 1 2 3
[1] 4 5 6
    \end{Verbatim}

    \begin{tcolorbox}[breakable, size=fbox, boxrule=1pt, pad at break*=1mm,colback=cellbackground, colframe=cellborder]
\prompt{In}{incolor}{31}{\boxspacing}
\begin{Verbatim}[commandchars=\\\{\}]
\PY{n}{v} \PY{o}{\PYZhy{}} \PY{n}{w}
\end{Verbatim}
\end{tcolorbox}

    \begin{enumerate*}
\item -3
\item -3
\item -3
\end{enumerate*}


    
    \begin{tcolorbox}[breakable, size=fbox, boxrule=1pt, pad at break*=1mm,colback=cellbackground, colframe=cellborder]
\prompt{In}{incolor}{32}{\boxspacing}
\begin{Verbatim}[commandchars=\\\{\}]
\PY{n}{v} \PY{o}{+} \PY{n}{w}
\end{Verbatim}
\end{tcolorbox}

    \begin{enumerate*}
\item 5
\item 7
\item 9
\end{enumerate*}


    
    \begin{tcolorbox}[breakable, size=fbox, boxrule=1pt, pad at break*=1mm,colback=cellbackground, colframe=cellborder]
\prompt{In}{incolor}{33}{\boxspacing}
\begin{Verbatim}[commandchars=\\\{\}]
\PY{n}{v} \PY{o}{*} \PY{n}{w}
\end{Verbatim}
\end{tcolorbox}

    \begin{enumerate*}
\item 4
\item 10
\item 18
\end{enumerate*}


    
    \begin{tcolorbox}[breakable, size=fbox, boxrule=1pt, pad at break*=1mm,colback=cellbackground, colframe=cellborder]
\prompt{In}{incolor}{34}{\boxspacing}
\begin{Verbatim}[commandchars=\\\{\}]
\PY{n}{v} \PY{o}{/} \PY{n}{w}
\end{Verbatim}
\end{tcolorbox}

    \begin{enumerate*}
\item 0.25
\item 0.4
\item 0.5
\end{enumerate*}


    
    Operator aritmatika:

\begin{itemize}
\tightlist
\item
  \texttt{+} : penjumlahan
\item
  \texttt{-} : pengurangan
\item
  \texttt{*} : perkalian
\item
  \texttt{\%\%} : sisa bagi (remainder)
\item
  \texttt{\^{}} : pemangkatan
\end{itemize}

    Operator relasi: * \texttt{\textless{}} : lebih kecil *
\texttt{\textgreater{}} : lebih besar * \texttt{==} : sama dengan
(operator kesamaan) * \texttt{\textless{}=} : lebih kecil sama dengan *
\texttt{\textgreater{}=} : lebih besar sama dengan * \texttt{!=} : tidak
sama

    \begin{tcolorbox}[breakable, size=fbox, boxrule=1pt, pad at break*=1mm,colback=cellbackground, colframe=cellborder]
\prompt{In}{incolor}{35}{\boxspacing}
\begin{Verbatim}[commandchars=\\\{\}]
\PY{n}{v} \PY{o}{\PYZlt{}\PYZhy{}} \PY{n+nf}{c}\PY{p}{(}\PY{l+m}{1}\PY{p}{,}\PY{l+m}{2}\PY{p}{,}\PY{l+m}{3}\PY{p}{)}
\PY{n}{w} \PY{o}{\PYZlt{}\PYZhy{}} \PY{l+m}{4}\PY{o}{:}\PY{l+m}{6}
\end{Verbatim}
\end{tcolorbox}

    \begin{tcolorbox}[breakable, size=fbox, boxrule=1pt, pad at break*=1mm,colback=cellbackground, colframe=cellborder]
\prompt{In}{incolor}{36}{\boxspacing}
\begin{Verbatim}[commandchars=\\\{\}]
\PY{n}{v} \PY{o}{\PYZgt{}} \PY{n}{w} \PY{c+c1}{\PYZsh{} bersifat element\PYZhy{}wise}
\end{Verbatim}
\end{tcolorbox}

    \begin{enumerate*}
\item FALSE
\item FALSE
\item FALSE
\end{enumerate*}


    
    \begin{tcolorbox}[breakable, size=fbox, boxrule=1pt, pad at break*=1mm,colback=cellbackground, colframe=cellborder]
\prompt{In}{incolor}{37}{\boxspacing}
\begin{Verbatim}[commandchars=\\\{\}]
\PY{n}{v} \PY{o}{!=} \PY{n}{w}
\end{Verbatim}
\end{tcolorbox}

    \begin{enumerate*}
\item TRUE
\item TRUE
\item TRUE
\end{enumerate*}


    
    \hypertarget{mengakses-elemen-vektor}{%
\subsubsection{Mengakses elemen vektor}\label{mengakses-elemen-vektor}}

    \begin{tcolorbox}[breakable, size=fbox, boxrule=1pt, pad at break*=1mm,colback=cellbackground, colframe=cellborder]
\prompt{In}{incolor}{38}{\boxspacing}
\begin{Verbatim}[commandchars=\\\{\}]
\PY{n}{A} \PY{o}{\PYZlt{}\PYZhy{}} \PY{n+nf}{seq}\PY{p}{(}\PY{l+m}{2}\PY{p}{,}\PY{l+m}{10}\PY{p}{,} \PY{n}{by}\PY{o}{=}\PY{l+m}{2}\PY{p}{)}
\PY{n}{A}
\end{Verbatim}
\end{tcolorbox}

    \begin{enumerate*}
\item 2
\item 4
\item 6
\item 8
\item 10
\end{enumerate*}


    
    \begin{tcolorbox}[breakable, size=fbox, boxrule=1pt, pad at break*=1mm,colback=cellbackground, colframe=cellborder]
\prompt{In}{incolor}{39}{\boxspacing}
\begin{Verbatim}[commandchars=\\\{\}]
\PY{n}{A}\PY{p}{[}\PY{n+nf}{c}\PY{p}{(}\PY{l+m}{1}\PY{p}{,}\PY{l+m}{2}\PY{p}{,}\PY{l+m}{3}\PY{p}{)}\PY{p}{]}
\end{Verbatim}
\end{tcolorbox}

    \begin{enumerate*}
\item 2
\item 4
\item 6
\end{enumerate*}


    
    \begin{tcolorbox}[breakable, size=fbox, boxrule=1pt, pad at break*=1mm,colback=cellbackground, colframe=cellborder]
\prompt{In}{incolor}{40}{\boxspacing}
\begin{Verbatim}[commandchars=\\\{\}]
\PY{n}{A}\PY{p}{[}\PY{l+m}{1}\PY{o}{:}\PY{l+m}{3}\PY{p}{]}
\end{Verbatim}
\end{tcolorbox}

    \begin{enumerate*}
\item 2
\item 4
\item 6
\end{enumerate*}


    
    \begin{tcolorbox}[breakable, size=fbox, boxrule=1pt, pad at break*=1mm,colback=cellbackground, colframe=cellborder]
\prompt{In}{incolor}{41}{\boxspacing}
\begin{Verbatim}[commandchars=\\\{\}]
\PY{n}{A}\PY{p}{[}\PY{n+nf}{c}\PY{p}{(}\PY{l+m}{\PYZhy{}1}\PY{p}{)}\PY{p}{]} \PY{c+c1}{\PYZsh{} tanpa indeks 1}
\end{Verbatim}
\end{tcolorbox}

    \begin{enumerate*}
\item 4
\item 6
\item 8
\item 10
\end{enumerate*}


    
    \begin{tcolorbox}[breakable, size=fbox, boxrule=1pt, pad at break*=1mm,colback=cellbackground, colframe=cellborder]
\prompt{In}{incolor}{42}{\boxspacing}
\begin{Verbatim}[commandchars=\\\{\}]
\PY{n}{A}\PY{p}{[}\PY{n+nf}{c}\PY{p}{(}\PY{l+m}{\PYZhy{}1}\PY{p}{,}\PY{l+m}{\PYZhy{}2}\PY{p}{)}\PY{p}{]} \PY{c+c1}{\PYZsh{} tanpa indeks 1 dan 2}
\end{Verbatim}
\end{tcolorbox}

    \begin{enumerate*}
\item 6
\item 8
\item 10
\end{enumerate*}


    
    \begin{tcolorbox}[breakable, size=fbox, boxrule=1pt, pad at break*=1mm,colback=cellbackground, colframe=cellborder]
\prompt{In}{incolor}{43}{\boxspacing}
\begin{Verbatim}[commandchars=\\\{\}]
\PY{n}{A}\PY{p}{[}\PY{n+nf}{c}\PY{p}{(}\PY{l+m}{\PYZhy{}1}\PY{p}{,}\PY{l+m}{\PYZhy{}2}\PY{p}{,}\PY{l+m}{\PYZhy{}3}\PY{p}{)}\PY{p}{]} \PY{c+c1}{\PYZsh{} tanpa indeks 1,2,dan 3}
\end{Verbatim}
\end{tcolorbox}

    \begin{enumerate*}
\item 8
\item 10
\end{enumerate*}


    
    \begin{tcolorbox}[breakable, size=fbox, boxrule=1pt, pad at break*=1mm,colback=cellbackground, colframe=cellborder]
\prompt{In}{incolor}{44}{\boxspacing}
\begin{Verbatim}[commandchars=\\\{\}]
\PY{n}{A}\PY{p}{[}\PY{n+nf}{c}\PY{p}{(}\PY{k+kc}{TRUE}\PY{p}{,} \PY{k+kc}{FALSE}\PY{p}{,} \PY{k+kc}{TRUE}\PY{p}{,} \PY{k+kc}{TRUE}\PY{p}{,} \PY{k+kc}{FALSE}\PY{p}{)}\PY{p}{]}
\end{Verbatim}
\end{tcolorbox}

    \begin{enumerate*}
\item 2
\item 6
\item 8
\end{enumerate*}


    
    \begin{itemize}
\item
  Kita dapat menggunakan \texttt{{[}{]}} untuk mengakses elemen vektor
  (\emph{indexing} dimulai dari \texttt{1}).
\item
  Nilai negatif digunakan untuk membuang elemen pada indeks yang tidak
  dikehendaki.
\item
  Nilai - nilai Boolean, \texttt{TRUE} dan \texttt{FALSE} juga dapat
  digunakan untuk pengindeksan vektor.
\end{itemize}

    \hypertarget{mendefinisikan-matriks}{%
\subsubsection{Mendefinisikan matriks}\label{mendefinisikan-matriks}}

    \[
A_{m,n} = 
\begin{pmatrix}
a_{1,1} & a_{1,2} & \cdots & a_{1,n} \\
a_{2,1} & a_{2,2} & \cdots & a_{2,n} \\
\vdots  & \vdots  & \ddots & \vdots  \\
a_{m,1} & a_{m,2} & \cdots & a_{m,n} 
\end{pmatrix}
\]

    Sintaks : \texttt{matrix(data,\ nrow,\ ncol,\ byrow,\ dimnames)}

\texttt{data}: elemen - elemen di dalam matriks,

\texttt{nrow}: jumlah baris,

\texttt{ncols}: jumlah kolom,

\texttt{byrow}: Jika \texttt{TRUE}, maka elemen - elemen matriks akan
disusun berdasarkan kolom

\texttt{dimnames}: Nama yg ditugaskan pada baris dan kolom.

contoh:

    \begin{tcolorbox}[breakable, size=fbox, boxrule=1pt, pad at break*=1mm,colback=cellbackground, colframe=cellborder]
\prompt{In}{incolor}{45}{\boxspacing}
\begin{Verbatim}[commandchars=\\\{\}]
\PY{n+nf}{matrix}\PY{p}{(}\PY{l+m}{1}\PY{o}{:}\PY{l+m}{9}\PY{p}{,} \PY{n}{nrow}\PY{o}{=}\PY{l+m}{3}\PY{p}{,} \PY{n}{ncol}\PY{o}{=}\PY{l+m}{3}\PY{p}{)}
\end{Verbatim}
\end{tcolorbox}

    A matrix: 3 × 3 of type int
\begin{tabular}{lll}
	 1 & 4 & 7\\
	 2 & 5 & 8\\
	 3 & 6 & 9\\
\end{tabular}


    
    \begin{tcolorbox}[breakable, size=fbox, boxrule=1pt, pad at break*=1mm,colback=cellbackground, colframe=cellborder]
\prompt{In}{incolor}{46}{\boxspacing}
\begin{Verbatim}[commandchars=\\\{\}]
\PY{n+nf}{matrix}\PY{p}{(}\PY{l+m}{1}\PY{o}{:}\PY{l+m}{9}\PY{p}{,} \PY{n}{nrow}\PY{o}{=}\PY{l+m}{3}\PY{p}{,} \PY{n}{ncol}\PY{o}{=}\PY{l+m}{3}\PY{p}{,} \PY{n}{byrow}\PY{o}{=}\PY{k+kc}{TRUE}\PY{p}{)}
\end{Verbatim}
\end{tcolorbox}

    A matrix: 3 × 3 of type int
\begin{tabular}{lll}
	 1 & 2 & 3\\
	 4 & 5 & 6\\
	 7 & 8 & 9\\
\end{tabular}


    
    \begin{tcolorbox}[breakable, size=fbox, boxrule=1pt, pad at break*=1mm,colback=cellbackground, colframe=cellborder]
\prompt{In}{incolor}{47}{\boxspacing}
\begin{Verbatim}[commandchars=\\\{\}]
\PY{n}{baris} \PY{o}{\PYZlt{}\PYZhy{}} \PY{n+nf}{c}\PY{p}{(}\PY{l+s}{\PYZsq{}}\PY{l+s}{baris1\PYZsq{}}\PY{p}{,} \PY{l+s}{\PYZsq{}}\PY{l+s}{baris2\PYZsq{}}\PY{p}{,} \PY{l+s}{\PYZsq{}}\PY{l+s}{baris3\PYZsq{}}\PY{p}{)}
\PY{n}{kolom} \PY{o}{\PYZlt{}\PYZhy{}} \PY{n+nf}{c}\PY{p}{(}\PY{l+s}{\PYZsq{}}\PY{l+s}{kolom1\PYZsq{}}\PY{p}{,} \PY{l+s}{\PYZsq{}}\PY{l+s}{kolom2\PYZsq{}}\PY{p}{,} \PY{l+s}{\PYZsq{}}\PY{l+s}{kolom3\PYZsq{}}\PY{p}{)}

\PY{n+nf}{matrix}\PY{p}{(}\PY{l+m}{1}\PY{o}{:}\PY{l+m}{9}\PY{p}{,} \PY{n}{nrow}\PY{o}{=}\PY{l+m}{3}\PY{p}{,} \PY{n}{ncol}\PY{o}{=}\PY{l+m}{3}\PY{p}{,} \PY{n}{byrow}\PY{o}{=}\PY{k+kc}{FALSE}\PY{p}{,} \PY{n}{dimnames} \PY{o}{=} \PY{n+nf}{list}\PY{p}{(}\PY{n}{baris}\PY{p}{,}\PY{n}{kolom}\PY{p}{)}\PY{p}{)}
\end{Verbatim}
\end{tcolorbox}

    A matrix: 3 × 3 of type int
\begin{tabular}{r|lll}
  & kolom1 & kolom2 & kolom3\\
\hline
	baris1 & 1 & 4 & 7\\
	baris2 & 2 & 5 & 8\\
	baris3 & 3 & 6 & 9\\
\end{tabular}


    
    \hypertarget{mengakses-elemen-matriks}{%
\subsubsection{Mengakses elemen
matriks}\label{mengakses-elemen-matriks}}

    \begin{tcolorbox}[breakable, size=fbox, boxrule=1pt, pad at break*=1mm,colback=cellbackground, colframe=cellborder]
\prompt{In}{incolor}{48}{\boxspacing}
\begin{Verbatim}[commandchars=\\\{\}]
\PY{n}{M} \PY{o}{=} \PY{n+nf}{matrix}\PY{p}{(}\PY{l+m}{1}\PY{o}{:}\PY{l+m}{9}\PY{p}{,} \PY{n}{nrow}\PY{o}{=}\PY{l+m}{3}\PY{p}{,} \PY{n}{ncol}\PY{o}{=}\PY{l+m}{3}\PY{p}{)}
\PY{n}{M}
\end{Verbatim}
\end{tcolorbox}

    A matrix: 3 × 3 of type int
\begin{tabular}{lll}
	 1 & 4 & 7\\
	 2 & 5 & 8\\
	 3 & 6 & 9\\
\end{tabular}


    
    Sintaks: \texttt{M{[}baris,\ kolom{]}}

    \begin{tcolorbox}[breakable, size=fbox, boxrule=1pt, pad at break*=1mm,colback=cellbackground, colframe=cellborder]
\prompt{In}{incolor}{49}{\boxspacing}
\begin{Verbatim}[commandchars=\\\{\}]
\PY{n}{M}\PY{p}{[}\PY{l+m}{1}\PY{p}{,}\PY{l+m}{1}\PY{p}{]}
\end{Verbatim}
\end{tcolorbox}

    1

    
    \begin{tcolorbox}[breakable, size=fbox, boxrule=1pt, pad at break*=1mm,colback=cellbackground, colframe=cellborder]
\prompt{In}{incolor}{50}{\boxspacing}
\begin{Verbatim}[commandchars=\\\{\}]
\PY{n}{M}\PY{p}{[}\PY{p}{,}\PY{l+m}{1}\PY{p}{]} \PY{c+c1}{\PYZsh{} kolom 1}
\end{Verbatim}
\end{tcolorbox}

    \begin{enumerate*}
\item 1
\item 2
\item 3
\end{enumerate*}


    
    \begin{tcolorbox}[breakable, size=fbox, boxrule=1pt, pad at break*=1mm,colback=cellbackground, colframe=cellborder]
\prompt{In}{incolor}{51}{\boxspacing}
\begin{Verbatim}[commandchars=\\\{\}]
\PY{n}{bar} \PY{o}{\PYZlt{}\PYZhy{}} \PY{n+nf}{c}\PY{p}{(}\PY{l+s}{\PYZsq{}}\PY{l+s}{b1\PYZsq{}}\PY{p}{,} \PY{l+s}{\PYZsq{}}\PY{l+s}{b2\PYZsq{}}\PY{p}{,} \PY{l+s}{\PYZsq{}}\PY{l+s}{b3\PYZsq{}}\PY{p}{)}
\PY{n}{kol} \PY{o}{\PYZlt{}\PYZhy{}} \PY{n+nf}{c}\PY{p}{(}\PY{l+s}{\PYZsq{}}\PY{l+s}{k1\PYZsq{}}\PY{p}{,} \PY{l+s}{\PYZsq{}}\PY{l+s}{k2\PYZsq{}}\PY{p}{,} \PY{l+s}{\PYZsq{}}\PY{l+s}{k3\PYZsq{}}\PY{p}{)}

\PY{n}{M} \PY{o}{\PYZlt{}\PYZhy{}} \PY{n+nf}{matrix}\PY{p}{(}\PY{l+m}{1}\PY{o}{:}\PY{l+m}{9}\PY{p}{,} \PY{n}{nrow}\PY{o}{=}\PY{l+m}{3}\PY{p}{,} \PY{n}{ncol}\PY{o}{=}\PY{l+m}{3}\PY{p}{,} \PY{n}{dimnames}\PY{o}{=}\PY{n+nf}{list}\PY{p}{(}\PY{n}{bar}\PY{p}{,}\PY{n}{kol}\PY{p}{)}\PY{p}{)}
\PY{n}{M}
\end{Verbatim}
\end{tcolorbox}

    A matrix: 3 × 3 of type int
\begin{tabular}{r|lll}
  & k1 & k2 & k3\\
\hline
	b1 & 1 & 4 & 7\\
	b2 & 2 & 5 & 8\\
	b3 & 3 & 6 & 9\\
\end{tabular}


    
    \begin{tcolorbox}[breakable, size=fbox, boxrule=1pt, pad at break*=1mm,colback=cellbackground, colframe=cellborder]
\prompt{In}{incolor}{52}{\boxspacing}
\begin{Verbatim}[commandchars=\\\{\}]
\PY{n}{M}\PY{p}{[}\PY{l+s}{\PYZsq{}}\PY{l+s}{b1\PYZsq{}}\PY{p}{,} \PY{p}{]} \PY{c+c1}{\PYZsh{} baris 1}
\end{Verbatim}
\end{tcolorbox}

    \begin{description*}
\item[k1] 1
\item[k2] 4
\item[k3] 7
\end{description*}


    
    \begin{tcolorbox}[breakable, size=fbox, boxrule=1pt, pad at break*=1mm,colback=cellbackground, colframe=cellborder]
\prompt{In}{incolor}{53}{\boxspacing}
\begin{Verbatim}[commandchars=\\\{\}]
\PY{n}{M}\PY{p}{[}\PY{l+m}{1}\PY{p}{,}\PY{p}{]}
\end{Verbatim}
\end{tcolorbox}

    \begin{description*}
\item[k1] 1
\item[k2] 4
\item[k3] 7
\end{description*}


    
    \hypertarget{dataframe}{%
\subsubsection{DataFrame}\label{dataframe}}

    DataFrame adalah suatu tabel atau struktur berbentuk menyerupai array
dua dimensi, di mana setiap kolomnya menyimpan data dari satu variabel
dan setiap barisnya menyimpan data untuk suatu ``titik'' data yang sama
dengan variabel yang berbeda - beda. Ibaratnya seperti spreadsheet MS
Excel.

    Sintaks: \texttt{data.frame(\#data)}

Contoh:

    \begin{tcolorbox}[breakable, size=fbox, boxrule=1pt, pad at break*=1mm,colback=cellbackground, colframe=cellborder]
\prompt{In}{incolor}{54}{\boxspacing}
\begin{Verbatim}[commandchars=\\\{\}]
\PY{n}{nama} \PY{o}{\PYZlt{}\PYZhy{}} \PY{n+nf}{c}\PY{p}{(}\PY{l+s}{\PYZsq{}}\PY{l+s}{Sandy\PYZsq{}}\PY{p}{,} \PY{l+s}{\PYZsq{}}\PY{l+s}{Evelyn\PYZsq{}}\PY{p}{,} \PY{l+s}{\PYZsq{}}\PY{l+s}{Brenda\PYZsq{}}\PY{p}{)}
\PY{n}{umur} \PY{o}{\PYZlt{}\PYZhy{}} \PY{n+nf}{c}\PY{p}{(}\PY{l+m}{43}\PY{p}{,} \PY{l+m}{23}\PY{p}{,} \PY{l+m}{45}\PY{p}{)}
\PY{n+nf}{data.frame}\PY{p}{(}\PY{n}{nama}\PY{p}{,}\PY{n}{umur}\PY{p}{)}
\end{Verbatim}
\end{tcolorbox}

    A data.frame: 3 × 2
\begin{tabular}{ll}
 nama & umur\\
 <fct> & <dbl>\\
\hline
	 Sandy  & 43\\
	 Evelyn & 23\\
	 Brenda & 45\\
\end{tabular}


    
    \hypertarget{mengakses-dataframe}{%
\paragraph{Mengakses DataFrame}\label{mengakses-dataframe}}

    \begin{tcolorbox}[breakable, size=fbox, boxrule=1pt, pad at break*=1mm,colback=cellbackground, colframe=cellborder]
\prompt{In}{incolor}{55}{\boxspacing}
\begin{Verbatim}[commandchars=\\\{\}]
\PY{n}{nama} \PY{o}{\PYZlt{}\PYZhy{}} \PY{n+nf}{c}\PY{p}{(}\PY{l+s}{\PYZsq{}}\PY{l+s}{Sandy\PYZsq{}}\PY{p}{,} \PY{l+s}{\PYZsq{}}\PY{l+s}{Evelyn\PYZsq{}}\PY{p}{,} \PY{l+s}{\PYZsq{}}\PY{l+s}{Brenda\PYZsq{}}\PY{p}{)}
\PY{n}{umur} \PY{o}{\PYZlt{}\PYZhy{}} \PY{n+nf}{c}\PY{p}{(}\PY{l+m}{43}\PY{p}{,} \PY{l+m}{23}\PY{p}{,} \PY{l+m}{45}\PY{p}{)}
\PY{n}{gaji} \PY{o}{\PYZlt{}\PYZhy{}} \PY{n+nf}{c}\PY{p}{(}\PY{l+m}{1000000}\PY{p}{,} \PY{l+m}{9000000}\PY{p}{,} \PY{l+m}{2000000}\PY{p}{)}

\PY{n}{D} \PY{o}{\PYZlt{}\PYZhy{}} \PY{n+nf}{data.frame}\PY{p}{(}\PY{n}{nama}\PY{p}{,}\PY{n}{umur}\PY{p}{,} \PY{n}{gaji}\PY{p}{)}
\PY{n}{D}
\end{Verbatim}
\end{tcolorbox}

    A data.frame: 3 × 3
\begin{tabular}{lll}
 nama & umur & gaji\\
 <fct> & <dbl> & <dbl>\\
\hline
	 Sandy  & 43 & 1e+06\\
	 Evelyn & 23 & 9e+06\\
	 Brenda & 45 & 2e+06\\
\end{tabular}


    
    Cara pertama:

    \begin{tcolorbox}[breakable, size=fbox, boxrule=1pt, pad at break*=1mm,colback=cellbackground, colframe=cellborder]
\prompt{In}{incolor}{56}{\boxspacing}
\begin{Verbatim}[commandchars=\\\{\}]
\PY{n}{D}\PY{o}{\PYZdl{}}\PY{n}{nama} \PY{c+c1}{\PYZsh{} akses kolom nama dari dataframe D}
\end{Verbatim}
\end{tcolorbox}

    \begin{enumerate*}
\item Sandy
\item Evelyn
\item Brenda
\end{enumerate*}

\emph{Levels}: \begin{enumerate*}
\item 'Brenda'
\item 'Evelyn'
\item 'Sandy'
\end{enumerate*}


    
    \begin{tcolorbox}[breakable, size=fbox, boxrule=1pt, pad at break*=1mm,colback=cellbackground, colframe=cellborder]
\prompt{In}{incolor}{57}{\boxspacing}
\begin{Verbatim}[commandchars=\\\{\}]
\PY{n}{D}\PY{o}{\PYZdl{}}\PY{n}{nama}\PY{p}{[}\PY{l+m}{1}\PY{p}{]} \PY{c+c1}{\PYZsh{} baris pertama dari nama}
\end{Verbatim}
\end{tcolorbox}

    Sandy
\emph{Levels}: \begin{enumerate*}
\item 'Brenda'
\item 'Evelyn'
\item 'Sandy'
\end{enumerate*}


    
    Cara kedua:

    \begin{tcolorbox}[breakable, size=fbox, boxrule=1pt, pad at break*=1mm,colback=cellbackground, colframe=cellborder]
\prompt{In}{incolor}{58}{\boxspacing}
\begin{Verbatim}[commandchars=\\\{\}]
\PY{n}{D}\PY{p}{[}\PY{l+m}{1}\PY{p}{,}\PY{p}{]} \PY{c+c1}{\PYZsh{} mengakses baris pertama}
\end{Verbatim}
\end{tcolorbox}

    A data.frame: 1 × 3
\begin{tabular}{r|lll}
  & nama & umur & gaji\\
  & <fct> & <dbl> & <dbl>\\
\hline
	1 & Sandy & 43 & 1e+06\\
\end{tabular}


    
    \begin{tcolorbox}[breakable, size=fbox, boxrule=1pt, pad at break*=1mm,colback=cellbackground, colframe=cellborder]
\prompt{In}{incolor}{59}{\boxspacing}
\begin{Verbatim}[commandchars=\\\{\}]
\PY{n}{D}\PY{p}{[}\PY{n+nf}{c}\PY{p}{(}\PY{l+m}{1}\PY{p}{,}\PY{l+m}{2}\PY{p}{)}\PY{p}{,}\PY{p}{]} \PY{c+c1}{\PYZsh{} akses seluruh kolom baris 1 \PYZhy{} 2}
\end{Verbatim}
\end{tcolorbox}

    A data.frame: 2 × 3
\begin{tabular}{r|lll}
  & nama & umur & gaji\\
  & <fct> & <dbl> & <dbl>\\
\hline
	1 & Sandy  & 43 & 1e+06\\
	2 & Evelyn & 23 & 9e+06\\
\end{tabular}


    
    \begin{tcolorbox}[breakable, size=fbox, boxrule=1pt, pad at break*=1mm,colback=cellbackground, colframe=cellborder]
\prompt{In}{incolor}{60}{\boxspacing}
\begin{Verbatim}[commandchars=\\\{\}]
\PY{n}{D}\PY{o}{\PYZdl{}}\PY{n}{gaji} \PY{o}{\PYZlt{}\PYZhy{}} \PY{n}{D}\PY{o}{\PYZdl{}}\PY{n}{gaji} \PY{o}{/}\PY{p}{(}\PY{l+m}{1e6}\PY{p}{)} \PY{c+c1}{\PYZsh{} membagi gaji sebanyak 1jt dan menugaskan ulang ke dataframe}
\PY{n}{D}
\end{Verbatim}
\end{tcolorbox}

    A data.frame: 3 × 3
\begin{tabular}{lll}
 nama & umur & gaji\\
 <fct> & <dbl> & <dbl>\\
\hline
	 Sandy  & 43 & 1\\
	 Evelyn & 23 & 9\\
	 Brenda & 45 & 2\\
\end{tabular}


    
    \begin{tcolorbox}[breakable, size=fbox, boxrule=1pt, pad at break*=1mm,colback=cellbackground, colframe=cellborder]
\prompt{In}{incolor}{61}{\boxspacing}
\begin{Verbatim}[commandchars=\\\{\}]
\PY{c+c1}{\PYZsh{} cara dapat intisari dataframe:}

\PY{n+nf}{summary}\PY{p}{(}\PY{n}{D}\PY{p}{)}
\end{Verbatim}
\end{tcolorbox}

    
    \begin{verbatim}
     nama        umur         gaji    
 Brenda:1   Min.   :23   Min.   :1.0  
 Evelyn:1   1st Qu.:33   1st Qu.:1.5  
 Sandy :1   Median :43   Median :2.0  
            Mean   :37   Mean   :4.0  
            3rd Qu.:44   3rd Qu.:5.5  
            Max.   :45   Max.   :9.0  
    \end{verbatim}

    
    \hypertarget{array}{%
\subsubsection{Array}\label{array}}

    Array merupakan koleksi elemen - elemen multidimensional yang mempunyai
tipe data seragam.

Sintaks: \texttt{array(data,\ dim,\ dimnames)}

Contoh:

    \begin{tcolorbox}[breakable, size=fbox, boxrule=1pt, pad at break*=1mm,colback=cellbackground, colframe=cellborder]
\prompt{In}{incolor}{62}{\boxspacing}
\begin{Verbatim}[commandchars=\\\{\}]
\PY{n+nf}{print}\PY{p}{(}\PY{n+nf}{array}\PY{p}{(}\PY{n}{data}\PY{o}{=}\PY{n+nf}{c}\PY{p}{(}\PY{l+s}{\PYZdq{}}\PY{l+s}{who\PYZdq{}}\PY{p}{,} \PY{l+s}{\PYZsq{}}\PY{l+s}{am\PYZsq{}}\PY{p}{,} \PY{l+s}{\PYZsq{}}\PY{l+s}{I\PYZsq{}}\PY{p}{)}\PY{p}{,} \PY{n}{dim} \PY{o}{=} \PY{n+nf}{c}\PY{p}{(}\PY{l+m}{3}\PY{p}{,}\PY{l+m}{3}\PY{p}{,}\PY{l+m}{2}\PY{p}{)}\PY{p}{)}\PY{p}{)}
\PY{c+c1}{\PYZsh{} dim =\PYZgt{} 3 baris, 3 kolom, 2 level (2 matriks)}
\end{Verbatim}
\end{tcolorbox}

    \begin{Verbatim}[commandchars=\\\{\}]
, , 1

     [,1]  [,2]  [,3]
[1,] "who" "who" "who"
[2,] "am"  "am"  "am"
[3,] "I"   "I"   "I"

, , 2

     [,1]  [,2]  [,3]
[1,] "who" "who" "who"
[2,] "am"  "am"  "am"
[3,] "I"   "I"   "I"

    \end{Verbatim}

    Harus diingat bahwa \emph{array} harus punya tipe data \textbf{YANG
SAMA}

    \begin{tcolorbox}[breakable, size=fbox, boxrule=1pt, pad at break*=1mm,colback=cellbackground, colframe=cellborder]
\prompt{In}{incolor}{63}{\boxspacing}
\begin{Verbatim}[commandchars=\\\{\}]
\PY{n}{bar} \PY{o}{\PYZlt{}\PYZhy{}} \PY{n+nf}{c}\PY{p}{(}\PY{l+s}{\PYZdq{}}\PY{l+s}{B1\PYZdq{}}\PY{p}{,} \PY{l+s}{\PYZdq{}}\PY{l+s}{B2\PYZdq{}}\PY{p}{,} \PY{l+s}{\PYZdq{}}\PY{l+s}{B3\PYZdq{}}\PY{p}{)}
\PY{n}{kol} \PY{o}{\PYZlt{}\PYZhy{}} \PY{n+nf}{c}\PY{p}{(}\PY{l+s}{\PYZdq{}}\PY{l+s}{K1\PYZdq{}}\PY{p}{,} \PY{l+s}{\PYZdq{}}\PY{l+s}{K2\PYZdq{}}\PY{p}{,} \PY{l+s}{\PYZdq{}}\PY{l+s}{K3\PYZdq{}}\PY{p}{)}
\PY{n}{mat} \PY{o}{\PYZlt{}\PYZhy{}} \PY{n+nf}{c}\PY{p}{(}\PY{l+s}{\PYZdq{}}\PY{l+s}{M1\PYZdq{}}\PY{p}{,} \PY{l+s}{\PYZdq{}}\PY{l+s}{M2\PYZdq{}}\PY{p}{)}

\PY{n+nf}{print}\PY{p}{(}\PY{n+nf}{array}\PY{p}{(}\PY{n}{data}\PY{o}{=}\PY{n+nf}{c}\PY{p}{(}\PY{l+s}{\PYZdq{}}\PY{l+s}{halo\PYZdq{}}\PY{p}{,}\PY{l+s}{\PYZdq{}}\PY{l+s}{hai\PYZdq{}}\PY{p}{,}\PY{l+s}{\PYZdq{}}\PY{l+s}{hoi\PYZdq{}}\PY{p}{)}\PY{p}{,} \PY{n}{dim}\PY{o}{=}\PY{n+nf}{c}\PY{p}{(}\PY{l+m}{3}\PY{p}{,}\PY{l+m}{3}\PY{p}{,}\PY{l+m}{2}\PY{p}{)}\PY{p}{,} \PY{n}{dimnames}\PY{o}{=}\PY{n+nf}{list}\PY{p}{(}\PY{n}{bar}\PY{p}{,}\PY{n}{kol}\PY{p}{,}\PY{n}{mat}\PY{p}{)}\PY{p}{)}\PY{p}{)}
\end{Verbatim}
\end{tcolorbox}

    \begin{Verbatim}[commandchars=\\\{\}]
, , M1

   K1     K2     K3
B1 "halo" "halo" "halo"
B2 "hai"  "hai"  "hai"
B3 "hoi"  "hoi"  "hoi"

, , M2

   K1     K2     K3
B1 "halo" "halo" "halo"
B2 "hai"  "hai"  "hai"
B3 "hoi"  "hoi"  "hoi"

    \end{Verbatim}

    \hypertarget{mengakses-array}{%
\paragraph{Mengakses array}\label{mengakses-array}}

    \begin{tcolorbox}[breakable, size=fbox, boxrule=1pt, pad at break*=1mm,colback=cellbackground, colframe=cellborder]
\prompt{In}{incolor}{64}{\boxspacing}
\begin{Verbatim}[commandchars=\\\{\}]
\PY{n}{bar} \PY{o}{\PYZlt{}\PYZhy{}} \PY{n+nf}{c}\PY{p}{(}\PY{l+s}{\PYZdq{}}\PY{l+s}{B1\PYZdq{}}\PY{p}{,} \PY{l+s}{\PYZdq{}}\PY{l+s}{B2\PYZdq{}}\PY{p}{,} \PY{l+s}{\PYZdq{}}\PY{l+s}{B3\PYZdq{}}\PY{p}{)}
\PY{n}{kol} \PY{o}{\PYZlt{}\PYZhy{}} \PY{n+nf}{c}\PY{p}{(}\PY{l+s}{\PYZdq{}}\PY{l+s}{K1\PYZdq{}}\PY{p}{,} \PY{l+s}{\PYZdq{}}\PY{l+s}{K2\PYZdq{}}\PY{p}{,} \PY{l+s}{\PYZdq{}}\PY{l+s}{K3\PYZdq{}}\PY{p}{)}
\PY{n}{mat} \PY{o}{\PYZlt{}\PYZhy{}} \PY{n+nf}{c}\PY{p}{(}\PY{l+s}{\PYZdq{}}\PY{l+s}{M1\PYZdq{}}\PY{p}{,} \PY{l+s}{\PYZdq{}}\PY{l+s}{M2\PYZdq{}}\PY{p}{)}

\PY{n}{A} \PY{o}{\PYZlt{}\PYZhy{}} \PY{n+nf}{array}\PY{p}{(}\PY{n}{data}\PY{o}{=}\PY{n+nf}{c}\PY{p}{(}\PY{l+s}{\PYZdq{}}\PY{l+s}{who\PYZdq{}}\PY{p}{,}\PY{l+s}{\PYZdq{}}\PY{l+s}{am\PYZdq{}}\PY{p}{,}\PY{l+s}{\PYZdq{}}\PY{l+s}{I\PYZdq{}}\PY{p}{)}\PY{p}{,} \PY{n}{dim}\PY{o}{=}\PY{n+nf}{c}\PY{p}{(}\PY{l+m}{3}\PY{p}{,}\PY{l+m}{3}\PY{p}{,}\PY{l+m}{2}\PY{p}{)}\PY{p}{,} \PY{n}{dimnames}\PY{o}{=}\PY{n+nf}{list}\PY{p}{(}\PY{n}{bar}\PY{p}{,}\PY{n}{kol}\PY{p}{,}\PY{n}{mat}\PY{p}{)}\PY{p}{)}
\PY{n+nf}{print}\PY{p}{(}\PY{n}{A}\PY{p}{)}
\end{Verbatim}
\end{tcolorbox}

    \begin{Verbatim}[commandchars=\\\{\}]
, , M1

   K1    K2    K3
B1 "who" "who" "who"
B2 "am"  "am"  "am"
B3 "I"   "I"   "I"

, , M2

   K1    K2    K3
B1 "who" "who" "who"
B2 "am"  "am"  "am"
B3 "I"   "I"   "I"

    \end{Verbatim}

    \begin{tcolorbox}[breakable, size=fbox, boxrule=1pt, pad at break*=1mm,colback=cellbackground, colframe=cellborder]
\prompt{In}{incolor}{65}{\boxspacing}
\begin{Verbatim}[commandchars=\\\{\}]
\PY{n}{A}\PY{p}{[}\PY{l+m}{1}\PY{p}{,}\PY{l+m}{1}\PY{p}{,}\PY{l+m}{1}\PY{p}{]} \PY{c+c1}{\PYZsh{} A[baris, kolom, matriks]}
\end{Verbatim}
\end{tcolorbox}

    'who'

    
    \begin{tcolorbox}[breakable, size=fbox, boxrule=1pt, pad at break*=1mm,colback=cellbackground, colframe=cellborder]
\prompt{In}{incolor}{66}{\boxspacing}
\begin{Verbatim}[commandchars=\\\{\}]
\PY{n}{A}\PY{p}{[}\PY{l+m}{1}\PY{p}{,}\PY{l+m}{1}\PY{p}{,}\PY{p}{]} \PY{c+c1}{\PYZsh{} baris 1, kolom 1, semua matriks}
\end{Verbatim}
\end{tcolorbox}

    \begin{description*}
\item[M1] 'who'
\item[M2] 'who'
\end{description*}


    
    \begin{tcolorbox}[breakable, size=fbox, boxrule=1pt, pad at break*=1mm,colback=cellbackground, colframe=cellborder]
\prompt{In}{incolor}{67}{\boxspacing}
\begin{Verbatim}[commandchars=\\\{\}]
\PY{n}{A}\PY{p}{[}\PY{l+m}{1}\PY{p}{,}\PY{p}{,}\PY{l+m}{1}\PY{p}{]} \PY{c+c1}{\PYZsh{} baris 1, semua kolom, matriks 1}
\end{Verbatim}
\end{tcolorbox}

    \begin{description*}
\item[K1] 'who'
\item[K2] 'who'
\item[K3] 'who'
\end{description*}


    
    \begin{tcolorbox}[breakable, size=fbox, boxrule=1pt, pad at break*=1mm,colback=cellbackground, colframe=cellborder]
\prompt{In}{incolor}{68}{\boxspacing}
\begin{Verbatim}[commandchars=\\\{\}]
\PY{n}{A}\PY{p}{[}\PY{l+s}{\PYZdq{}}\PY{l+s}{B1\PYZdq{}}\PY{p}{,}\PY{p}{,}\PY{p}{]} \PY{c+c1}{\PYZsh{} baris B1, semua kolom, semua matriks}
\end{Verbatim}
\end{tcolorbox}

    A matrix: 3 × 2 of type chr
\begin{tabular}{r|ll}
  & M1 & M2\\
\hline
	K1 & who & who\\
	K2 & who & who\\
	K3 & who & who\\
\end{tabular}


    
    \hypertarget{mengenal-list}{%
\subsubsection{Mengenal list}\label{mengenal-list}}

    \begin{tcolorbox}[breakable, size=fbox, boxrule=1pt, pad at break*=1mm,colback=cellbackground, colframe=cellborder]
\prompt{In}{incolor}{69}{\boxspacing}
\begin{Verbatim}[commandchars=\\\{\}]
\PY{n}{L} \PY{o}{\PYZlt{}\PYZhy{}} \PY{n+nf}{list}\PY{p}{(}\PY{l+s}{\PYZdq{}}\PY{l+s}{PNP\PYZdq{}}\PY{p}{,} \PY{l+m}{1}\PY{o}{:}\PY{l+m}{3}\PY{p}{,} \PY{k+kc}{TRUE}\PY{p}{,} \PY{n+nf}{matrix}\PY{p}{(}\PY{l+m}{1}\PY{o}{:}\PY{l+m}{6}\PY{p}{,} \PY{n}{nrow}\PY{o}{=}\PY{l+m}{2}\PY{p}{)}\PY{p}{,} \PY{n+nf}{array}\PY{p}{(}\PY{l+m}{5}\PY{o}{:}\PY{l+m}{10}\PY{p}{,} \PY{n}{dim} \PY{o}{=} \PY{n+nf}{c}\PY{p}{(}\PY{l+m}{3}\PY{p}{,}\PY{l+m}{3}\PY{p}{,}\PY{l+m}{2}\PY{p}{)}\PY{p}{)}\PY{p}{)}
\PY{n+nf}{print}\PY{p}{(}\PY{n}{L}\PY{p}{)}
\end{Verbatim}
\end{tcolorbox}

    \begin{Verbatim}[commandchars=\\\{\}]
[[1]]
[1] "PNP"

[[2]]
[1] 1 2 3

[[3]]
[1] TRUE

[[4]]
     [,1] [,2] [,3]
[1,]    1    3    5
[2,]    2    4    6

[[5]]
, , 1

     [,1] [,2] [,3]
[1,]    5    8    5
[2,]    6    9    6
[3,]    7   10    7

, , 2

     [,1] [,2] [,3]
[1,]    8    5    8
[2,]    9    6    9
[3,]   10    7   10


    \end{Verbatim}

    \texttt{L} memuat: karakter, vektor, boolean, matriks, array

    List adalah koleksi berbagai macam struktur data

    \hypertarget{mengenal-factor}{%
\subsubsection{Mengenal factor}\label{mengenal-factor}}

    \begin{itemize}
\item
  Digunakan untuk menyelesaikan permasalahan pengolahan data
  \emph{string} yang seringkali bersifat \emph{memory intensive}.
\item
  \emph{Factor} meng-\emph{encode} \emph{string} menjadi nilai numerik.
\item
  Baik untuk optimisasi algoritma.
\end{itemize}

    \begin{tcolorbox}[breakable, size=fbox, boxrule=1pt, pad at break*=1mm,colback=cellbackground, colframe=cellborder]
\prompt{In}{incolor}{70}{\boxspacing}
\begin{Verbatim}[commandchars=\\\{\}]
\PY{n}{jenisKel} \PY{o}{\PYZlt{}\PYZhy{}} \PY{n+nf}{c}\PY{p}{(}\PY{l+s}{\PYZdq{}}\PY{l+s}{Pria\PYZdq{}}\PY{p}{,} \PY{l+s}{\PYZdq{}}\PY{l+s}{Wanita\PYZdq{}}\PY{p}{)}
\PY{n}{factorGen} \PY{o}{\PYZlt{}\PYZhy{}} \PY{n+nf}{factor}\PY{p}{(}\PY{n}{jenisKel}\PY{p}{)}
\PY{n+nf}{print}\PY{p}{(}\PY{n}{factorGen}\PY{p}{)}
\end{Verbatim}
\end{tcolorbox}

    \begin{Verbatim}[commandchars=\\\{\}]
[1] Pria   Wanita
Levels: Pria Wanita
    \end{Verbatim}

    \begin{tcolorbox}[breakable, size=fbox, boxrule=1pt, pad at break*=1mm,colback=cellbackground, colframe=cellborder]
\prompt{In}{incolor}{71}{\boxspacing}
\begin{Verbatim}[commandchars=\\\{\}]
\PY{n+nf}{str}\PY{p}{(}\PY{n}{factorGen}\PY{p}{)} \PY{c+c1}{\PYZsh{} ada dua level: Pria dan Wanita}
\end{Verbatim}
\end{tcolorbox}

    \begin{Verbatim}[commandchars=\\\{\}]
 Factor w/ 2 levels "Pria","Wanita": 1 2
    \end{Verbatim}

    \hypertarget{operator-penugasan}{%
\subsection{Operator Penugasan}\label{operator-penugasan}}

    \begin{tcolorbox}[breakable, size=fbox, boxrule=1pt, pad at break*=1mm,colback=cellbackground, colframe=cellborder]
\prompt{In}{incolor}{72}{\boxspacing}
\begin{Verbatim}[commandchars=\\\{\}]
\PY{c+c1}{\PYZsh{} Kedua operator penugasan ini dapat digunakan}

\PY{n}{x1} \PY{o}{=} \PY{l+m}{13}
\PY{n}{x2} \PY{o}{\PYZlt{}\PYZhy{}} \PY{l+m}{13}

\PY{n}{x1} \PY{o}{==} \PY{n}{x2}
\end{Verbatim}
\end{tcolorbox}

    TRUE

    
    Namun harus hati - hati juga tidak setiap saat kedua operator ini dapat
berlaku dengan sama. Misalnya:

    \begin{tcolorbox}[breakable, size=fbox, boxrule=1pt, pad at break*=1mm,colback=cellbackground, colframe=cellborder]
\prompt{In}{incolor}{73}{\boxspacing}
\begin{Verbatim}[commandchars=\\\{\}]
\PY{n+nf}{mean}\PY{p}{(}\PY{n}{x} \PY{o}{=} \PY{l+m}{1}\PY{o}{:}\PY{l+m}{10}\PY{p}{)}
\end{Verbatim}
\end{tcolorbox}

    5.5

    
    \begin{tcolorbox}[breakable, size=fbox, boxrule=1pt, pad at break*=1mm,colback=cellbackground, colframe=cellborder]
\prompt{In}{incolor}{74}{\boxspacing}
\begin{Verbatim}[commandchars=\\\{\}]
\PY{n}{x} \PY{c+c1}{\PYZsh{} variabel x hanya dapat diakses di dalam fungsi tsb (local var) (x bukan variabel yg sama)}
\end{Verbatim}
\end{tcolorbox}

    \begin{enumerate*}
\item 1
\item 1.77777777777778
\item 2.55555555555556
\item 3.33333333333333
\item 4.11111111111111
\item 4.88888888888889
\item 5.66666666666667
\item 6.44444444444444
\item 7.22222222222222
\item 8
\end{enumerate*}


    
    \begin{tcolorbox}[breakable, size=fbox, boxrule=1pt, pad at break*=1mm,colback=cellbackground, colframe=cellborder]
\prompt{In}{incolor}{75}{\boxspacing}
\begin{Verbatim}[commandchars=\\\{\}]
\PY{n+nf}{mean}\PY{p}{(}\PY{n}{x} \PY{o}{\PYZlt{}\PYZhy{}} \PY{l+m}{1}\PY{o}{:}\PY{l+m}{10}\PY{p}{)}
\end{Verbatim}
\end{tcolorbox}

    5.5

    
    \begin{tcolorbox}[breakable, size=fbox, boxrule=1pt, pad at break*=1mm,colback=cellbackground, colframe=cellborder]
\prompt{In}{incolor}{76}{\boxspacing}
\begin{Verbatim}[commandchars=\\\{\}]
\PY{n}{x} \PY{c+c1}{\PYZsh{} variabel x bersifat global}
\end{Verbatim}
\end{tcolorbox}

    \begin{enumerate*}
\item 1
\item 2
\item 3
\item 4
\item 5
\item 6
\item 7
\item 8
\item 9
\item 10
\end{enumerate*}


    
    Jika kita menggunakan \texttt{\textless{}-}, maka nilai tsb akan menjadi
variabel global, sementara \texttt{=} hanya bersifat lokal.

    \hypertarget{eksplorasi-data}{%
\section{Eksplorasi data}\label{eksplorasi-data}}

    Kita akan menggunakan Iris dataset dari situs
\href{https://archive.ics.uci.edu/ml/datasets/iris}{University of
California Irvine Machine Learning Repository}.

Hal - hal penting yang harus kita catat berkaitan dengan data ini adalah
sebagai berikut:

Attribute Information:

\begin{enumerate}
\def\labelenumi{\arabic{enumi}.}
\tightlist
\item
  sepal length in cm
\item
  sepal width in cm
\item
  petal length in cm
\item
  petal width in cm
\item
  class: * Iris Setosa * Iris Versicolour * Iris Virginica
\end{enumerate}

    \hypertarget{pengolahan-data}{%
\subsection{Pengolahan data}\label{pengolahan-data}}

    \hypertarget{mengimpor-data}{%
\subsubsection{Mengimpor data}\label{mengimpor-data}}

    \hypertarget{cara-1-menggunakan-tautan-httpsarchive.ics.uci.edumlmachine-learning-databasesirisiris.data}{%
\paragraph{Cara 1: Menggunakan tautan :
``https://archive.ics.uci.edu/ml/machine-learning-databases/iris/iris.data''}\label{cara-1-menggunakan-tautan-httpsarchive.ics.uci.edumlmachine-learning-databasesirisiris.data}}

    \begin{tcolorbox}[breakable, size=fbox, boxrule=1pt, pad at break*=1mm,colback=cellbackground, colframe=cellborder]
\prompt{In}{incolor}{77}{\boxspacing}
\begin{Verbatim}[commandchars=\\\{\}]
\PY{n}{tautan} \PY{o}{\PYZlt{}\PYZhy{}} \PY{l+s}{\PYZdq{}}\PY{l+s}{https://archive.ics.uci.edu/ml/machine\PYZhy{}learning\PYZhy{}databases/iris/iris.data\PYZdq{}}

\PY{n}{df} \PY{o}{\PYZlt{}\PYZhy{}} \PY{n+nf}{read.csv}\PY{p}{(}\PY{n+nf}{url}\PY{p}{(}\PY{n}{tautan}\PY{p}{)}\PY{p}{,} \PY{n}{header}\PY{o}{=}\PY{k+kc}{FALSE}\PY{p}{)} \PY{c+c1}{\PYZsh{} header nya tidak mau kita ikut sertakan}
\PY{n+nf}{head}\PY{p}{(}\PY{n}{df}\PY{p}{)} \PY{c+c1}{\PYZsh{} 6 baris pertama}
\end{Verbatim}
\end{tcolorbox}

    A data.frame: 6 × 5
\begin{tabular}{r|lllll}
  & V1 & V2 & V3 & V4 & V5\\
  & <dbl> & <dbl> & <dbl> & <dbl> & <fct>\\
\hline
	1 & 5.1 & 3.5 & 1.4 & 0.2 & Iris-setosa\\
	2 & 4.9 & 3.0 & 1.4 & 0.2 & Iris-setosa\\
	3 & 4.7 & 3.2 & 1.3 & 0.2 & Iris-setosa\\
	4 & 4.6 & 3.1 & 1.5 & 0.2 & Iris-setosa\\
	5 & 5.0 & 3.6 & 1.4 & 0.2 & Iris-setosa\\
	6 & 5.4 & 3.9 & 1.7 & 0.4 & Iris-setosa\\
\end{tabular}


    
    \hypertarget{cara-2-menggunakan-pustaka}{%
\paragraph{Cara 2: Menggunakan
pustaka}\label{cara-2-menggunakan-pustaka}}

    \begin{tcolorbox}[breakable, size=fbox, boxrule=1pt, pad at break*=1mm,colback=cellbackground, colframe=cellborder]
\prompt{In}{incolor}{78}{\boxspacing}
\begin{Verbatim}[commandchars=\\\{\}]
\PY{n+nf}{library}\PY{p}{(}\PY{n}{datasets}\PY{p}{)} \PY{c+c1}{\PYZsh{} iris termasuk ke dalam datasets pustaka ini}
\PY{n}{df} \PY{o}{\PYZlt{}\PYZhy{}} \PY{n}{iris} \PY{c+c1}{\PYZsh{} langsung terimpor ke workspace kita}
\PY{n+nf}{head}\PY{p}{(}\PY{n}{df}\PY{p}{)}
\end{Verbatim}
\end{tcolorbox}

    A data.frame: 6 × 5
\begin{tabular}{r|lllll}
  & Sepal.Length & Sepal.Width & Petal.Length & Petal.Width & Species\\
  & <dbl> & <dbl> & <dbl> & <dbl> & <fct>\\
\hline
	1 & 5.1 & 3.5 & 1.4 & 0.2 & setosa\\
	2 & 4.9 & 3.0 & 1.4 & 0.2 & setosa\\
	3 & 4.7 & 3.2 & 1.3 & 0.2 & setosa\\
	4 & 4.6 & 3.1 & 1.5 & 0.2 & setosa\\
	5 & 5.0 & 3.6 & 1.4 & 0.2 & setosa\\
	6 & 5.4 & 3.9 & 1.7 & 0.4 & setosa\\
\end{tabular}


    
    \hypertarget{cara-3-mengunduh-file-secara-lokal}{%
\paragraph{Cara 3: Mengunduh file secara
lokal}\label{cara-3-mengunduh-file-secara-lokal}}

    Pergi ke
https://archive.ics.uci.edu/ml/machine-learning-databases/iris/iris.data,
unduh filenya, pindahkan ke direktori tempat kalian bekerja (atau di
mana pun juga boleh yang penting kalian tahu PATH -nya), jalankan:

    \begin{tcolorbox}[breakable, size=fbox, boxrule=1pt, pad at break*=1mm,colback=cellbackground, colframe=cellborder]
\prompt{In}{incolor}{79}{\boxspacing}
\begin{Verbatim}[commandchars=\\\{\}]
\PY{n}{df} \PY{o}{\PYZlt{}\PYZhy{}} \PY{n+nf}{read.csv}\PY{p}{(}\PY{l+s}{\PYZdq{}}\PY{l+s}{./notebooks/iris.data\PYZdq{}}\PY{p}{,} \PY{n}{header}\PY{o}{=}\PY{k+kc}{FALSE}\PY{p}{)}
\PY{n+nf}{head}\PY{p}{(}\PY{n}{df}\PY{p}{)}
\end{Verbatim}
\end{tcolorbox}

    A data.frame: 6 × 5
\begin{tabular}{r|lllll}
  & V1 & V2 & V3 & V4 & V5\\
  & <dbl> & <dbl> & <dbl> & <dbl> & <fct>\\
\hline
	1 & 5.1 & 3.5 & 1.4 & 0.2 & Iris-setosa\\
	2 & 4.9 & 3.0 & 1.4 & 0.2 & Iris-setosa\\
	3 & 4.7 & 3.2 & 1.3 & 0.2 & Iris-setosa\\
	4 & 4.6 & 3.1 & 1.5 & 0.2 & Iris-setosa\\
	5 & 5.0 & 3.6 & 1.4 & 0.2 & Iris-setosa\\
	6 & 5.4 & 3.9 & 1.7 & 0.4 & Iris-setosa\\
\end{tabular}


    
    \hypertarget{menyiapkan-data}{%
\subsubsection{Menyiapkan data}\label{menyiapkan-data}}

    \hypertarget{mengetahui-dimensi-dataframe}{%
\paragraph{Mengetahui dimensi
DataFrame}\label{mengetahui-dimensi-dataframe}}

    \begin{tcolorbox}[breakable, size=fbox, boxrule=1pt, pad at break*=1mm,colback=cellbackground, colframe=cellborder]
\prompt{In}{incolor}{80}{\boxspacing}
\begin{Verbatim}[commandchars=\\\{\}]
\PY{n+nf}{dim}\PY{p}{(}\PY{n}{df}\PY{p}{)} \PY{c+c1}{\PYZsh{} baris = 150, kolom = 5}
\end{Verbatim}
\end{tcolorbox}

    \begin{enumerate*}
\item 150
\item 5
\end{enumerate*}


    
    \hypertarget{menamakan-kolom}{%
\paragraph{Menamakan kolom}\label{menamakan-kolom}}

    \begin{tcolorbox}[breakable, size=fbox, boxrule=1pt, pad at break*=1mm,colback=cellbackground, colframe=cellborder]
\prompt{In}{incolor}{81}{\boxspacing}
\begin{Verbatim}[commandchars=\\\{\}]
\PY{n+nf}{tail}\PY{p}{(}\PY{n}{df}\PY{p}{)} \PY{c+c1}{\PYZsh{} belum mempunyai nama kolom}
\end{Verbatim}
\end{tcolorbox}

    A data.frame: 6 × 5
\begin{tabular}{r|lllll}
  & V1 & V2 & V3 & V4 & V5\\
  & <dbl> & <dbl> & <dbl> & <dbl> & <fct>\\
\hline
	145 & 6.7 & 3.3 & 5.7 & 2.5 & Iris-virginica\\
	146 & 6.7 & 3.0 & 5.2 & 2.3 & Iris-virginica\\
	147 & 6.3 & 2.5 & 5.0 & 1.9 & Iris-virginica\\
	148 & 6.5 & 3.0 & 5.2 & 2.0 & Iris-virginica\\
	149 & 6.2 & 3.4 & 5.4 & 2.3 & Iris-virginica\\
	150 & 5.9 & 3.0 & 5.1 & 1.8 & Iris-virginica\\
\end{tabular}


    
    \begin{tcolorbox}[breakable, size=fbox, boxrule=1pt, pad at break*=1mm,colback=cellbackground, colframe=cellborder]
\prompt{In}{incolor}{82}{\boxspacing}
\begin{Verbatim}[commandchars=\\\{\}]
\PY{n+nf}{colnames}\PY{p}{(}\PY{n}{df}\PY{p}{)} \PY{o}{=} \PY{n+nf}{c}\PY{p}{(}\PY{l+s}{\PYZdq{}}\PY{l+s}{Sepal.Length\PYZdq{}}\PY{p}{,} \PY{l+s}{\PYZdq{}}\PY{l+s}{Sepal.Width\PYZdq{}}\PY{p}{,} \PY{l+s}{\PYZdq{}}\PY{l+s}{Petal.Length\PYZdq{}}\PY{p}{,} \PY{l+s}{\PYZdq{}}\PY{l+s}{Petal.Width\PYZdq{}}\PY{p}{,} \PY{l+s}{\PYZdq{}}\PY{l+s}{Species\PYZdq{}}\PY{p}{)}
\PY{n+nf}{head}\PY{p}{(}\PY{n}{df}\PY{p}{)}
\end{Verbatim}
\end{tcolorbox}

    A data.frame: 6 × 5
\begin{tabular}{r|lllll}
  & Sepal.Length & Sepal.Width & Petal.Length & Petal.Width & Species\\
  & <dbl> & <dbl> & <dbl> & <dbl> & <fct>\\
\hline
	1 & 5.1 & 3.5 & 1.4 & 0.2 & Iris-setosa\\
	2 & 4.9 & 3.0 & 1.4 & 0.2 & Iris-setosa\\
	3 & 4.7 & 3.2 & 1.3 & 0.2 & Iris-setosa\\
	4 & 4.6 & 3.1 & 1.5 & 0.2 & Iris-setosa\\
	5 & 5.0 & 3.6 & 1.4 & 0.2 & Iris-setosa\\
	6 & 5.4 & 3.9 & 1.7 & 0.4 & Iris-setosa\\
\end{tabular}


    
    \hypertarget{mengetahui-nilai-kosong}{%
\paragraph{Mengetahui nilai kosong}\label{mengetahui-nilai-kosong}}

    Dataset Iris ini contoh data yang sempurna, karena tidak terdapat nilai
kosong, namun sebagian besar data yang kita temui di dunia nyata
tidaklah demikian. Akan banyak sekali nilai kosongnya. Ada banyak sekali
metode yang digunakan untuk menangani nilai - nilai kosong (mis.
mengganti dengan nilai rata - rata, dsb), namun karena fokus kita pada
tutorial ini hanya mengangani dataset iris, maka kita tidak akan
membahas topik penanganan nilai kosong ini.

    Nilai kosong di R ditandai dengan nilai \texttt{NA}. Untuk mengetahui
keberadaan nilai kosong, gunakan:

    \begin{tcolorbox}[breakable, size=fbox, boxrule=1pt, pad at break*=1mm,colback=cellbackground, colframe=cellborder]
\prompt{In}{incolor}{112}{\boxspacing}
\begin{Verbatim}[commandchars=\\\{\}]
\PY{n+nf}{head}\PY{p}{(}\PY{n+nf}{is.na}\PY{p}{(}\PY{n}{df}\PY{p}{)}\PY{p}{)} \PY{c+c1}{\PYZsh{} seluruh entri FALSE: ga ada nilai kosong}
\end{Verbatim}
\end{tcolorbox}

    A matrix: 6 × 5 of type lgl
\begin{tabular}{lllll}
 Sepal.Length & Sepal.Width & Petal.Length & Petal.Width & Species\\
\hline
	 FALSE & FALSE & FALSE & FALSE & FALSE\\
	 FALSE & FALSE & FALSE & FALSE & FALSE\\
	 FALSE & FALSE & FALSE & FALSE & FALSE\\
	 FALSE & FALSE & FALSE & FALSE & FALSE\\
	 FALSE & FALSE & FALSE & FALSE & FALSE\\
	 FALSE & FALSE & FALSE & FALSE & FALSE\\
\end{tabular}


    
    Tapi hal di atas susah dilihat mata, maka kita gunakan:

    \begin{tcolorbox}[breakable, size=fbox, boxrule=1pt, pad at break*=1mm,colback=cellbackground, colframe=cellborder]
\prompt{In}{incolor}{84}{\boxspacing}
\begin{Verbatim}[commandchars=\\\{\}]
\PY{n+nf}{any}\PY{p}{(}\PY{n+nf}{is.na}\PY{p}{(}\PY{n}{df}\PY{p}{)}\PY{p}{)} 
\PY{c+c1}{\PYZsh{} buat mengecek ke seluruh nilai ada ga yg kosong kalo ada 1 aja maka nilainya jd TRUE}
\end{Verbatim}
\end{tcolorbox}

    FALSE

    
    Kalau mau tahu berapa jumlah nilai \texttt{NA} di dataset, gunakan:

    \begin{tcolorbox}[breakable, size=fbox, boxrule=1pt, pad at break*=1mm,colback=cellbackground, colframe=cellborder]
\prompt{In}{incolor}{85}{\boxspacing}
\begin{Verbatim}[commandchars=\\\{\}]
\PY{n+nf}{sum}\PY{p}{(}\PY{n+nf}{is.na}\PY{p}{(}\PY{n}{df}\PY{p}{)}\PY{p}{)}
\end{Verbatim}
\end{tcolorbox}

    0

    
    Berikut ini contoh dataframe yang terdapat nilai kosongnya:

    \begin{tcolorbox}[breakable, size=fbox, boxrule=1pt, pad at break*=1mm,colback=cellbackground, colframe=cellborder]
\prompt{In}{incolor}{86}{\boxspacing}
\begin{Verbatim}[commandchars=\\\{\}]
\PY{n}{df1} \PY{o}{=} \PY{n+nf}{data.frame}\PY{p}{(}\PY{l+m}{1}\PY{o}{:}\PY{l+m}{4}\PY{p}{,} \PY{n+nf}{c}\PY{p}{(}\PY{l+m}{6}\PY{p}{,}\PY{l+m}{7}\PY{p}{,}\PY{k+kc}{NA}\PY{p}{,}\PY{k+kc}{NA}\PY{p}{)}\PY{p}{)}
\PY{n}{df1}
\end{Verbatim}
\end{tcolorbox}

    A data.frame: 4 × 2
\begin{tabular}{ll}
 X1.4 & c.6..7..NA..NA.\\
 <int> & <dbl>\\
\hline
	 1 &  6\\
	 2 &  7\\
	 3 & NA\\
	 4 & NA\\
\end{tabular}


    
    \begin{tcolorbox}[breakable, size=fbox, boxrule=1pt, pad at break*=1mm,colback=cellbackground, colframe=cellborder]
\prompt{In}{incolor}{87}{\boxspacing}
\begin{Verbatim}[commandchars=\\\{\}]
\PY{n+nf}{is.na}\PY{p}{(}\PY{n}{df1}\PY{p}{)}
\end{Verbatim}
\end{tcolorbox}

    A matrix: 4 × 2 of type lgl
\begin{tabular}{ll}
 X1.4 & c.6..7..NA..NA.\\
\hline
	 FALSE & FALSE\\
	 FALSE & FALSE\\
	 FALSE &  TRUE\\
	 FALSE &  TRUE\\
\end{tabular}


    
    \begin{tcolorbox}[breakable, size=fbox, boxrule=1pt, pad at break*=1mm,colback=cellbackground, colframe=cellborder]
\prompt{In}{incolor}{88}{\boxspacing}
\begin{Verbatim}[commandchars=\\\{\}]
\PY{n+nf}{any}\PY{p}{(}\PY{n+nf}{is.na}\PY{p}{(}\PY{n}{df1}\PY{p}{)}\PY{p}{)}
\end{Verbatim}
\end{tcolorbox}

    TRUE

    
    \begin{tcolorbox}[breakable, size=fbox, boxrule=1pt, pad at break*=1mm,colback=cellbackground, colframe=cellborder]
\prompt{In}{incolor}{89}{\boxspacing}
\begin{Verbatim}[commandchars=\\\{\}]
\PY{n+nf}{sum}\PY{p}{(}\PY{n+nf}{is.na}\PY{p}{(}\PY{n}{df1}\PY{p}{)}\PY{p}{)}
\end{Verbatim}
\end{tcolorbox}

    2

    
    \hypertarget{visualisasi-data}{%
\subsection{Visualisasi data}\label{visualisasi-data}}

    Kita akan menggunakan pustaka graphics untuk visualisasi dasar (bukan
menggunakan ggplot2).

    \begin{tcolorbox}[breakable, size=fbox, boxrule=1pt, pad at break*=1mm,colback=cellbackground, colframe=cellborder]
\prompt{In}{incolor}{90}{\boxspacing}
\begin{Verbatim}[commandchars=\\\{\}]
\PY{n+nf}{library}\PY{p}{(}\PY{n}{graphics}\PY{p}{)}
\end{Verbatim}
\end{tcolorbox}

    Setiap kali hendak menggunakan pustaka baru, diharapkan kalian
mengunjungi situs:

https://www.rdocumentation.org/

Untuk mencari petunjuk penggunaannya. Karena keunggulan R (bagi kami
pribadi dibandingkan Python) adalah kelengkapan dokumentasinya (karena
memang diciptakan untuk tujuan akademik???).

    \hypertarget{scatterplot}{%
\subsubsection{Scatterplot}\label{scatterplot}}

    \begin{tcolorbox}[breakable, size=fbox, boxrule=1pt, pad at break*=1mm,colback=cellbackground, colframe=cellborder]
\prompt{In}{incolor}{91}{\boxspacing}
\begin{Verbatim}[commandchars=\\\{\}]
\PY{n}{x} \PY{o}{\PYZlt{}\PYZhy{}} \PY{l+m}{\PYZhy{}4}\PY{o}{:}\PY{l+m}{4}
\PY{n}{y} \PY{o}{\PYZlt{}\PYZhy{}} \PY{n}{x}\PY{o}{\PYZca{}}\PY{l+m}{2}

\PY{n+nf}{plot}\PY{p}{(}\PY{n}{x}\PY{p}{,}\PY{n}{y}\PY{p}{)}
\end{Verbatim}
\end{tcolorbox}

    \begin{center}
    \adjustimage{max size={0.9\linewidth}{0.9\paperheight}}{output_160_0.png}
    \end{center}
    { \hspace*{\fill} \\}
    
    \begin{tcolorbox}[breakable, size=fbox, boxrule=1pt, pad at break*=1mm,colback=cellbackground, colframe=cellborder]
\prompt{In}{incolor}{92}{\boxspacing}
\begin{Verbatim}[commandchars=\\\{\}]
\PY{n+nf}{plot}\PY{p}{(}\PY{n}{x}\PY{p}{,}\PY{n}{y}\PY{p}{,} \PY{n}{type}\PY{o}{=}\PY{l+s}{\PYZdq{}}\PY{l+s}{l\PYZdq{}}\PY{p}{)}
\end{Verbatim}
\end{tcolorbox}

    \begin{center}
    \adjustimage{max size={0.9\linewidth}{0.9\paperheight}}{output_161_0.png}
    \end{center}
    { \hspace*{\fill} \\}
    
    \begin{tcolorbox}[breakable, size=fbox, boxrule=1pt, pad at break*=1mm,colback=cellbackground, colframe=cellborder]
\prompt{In}{incolor}{93}{\boxspacing}
\begin{Verbatim}[commandchars=\\\{\}]
\PY{n+nf}{plot}\PY{p}{(}\PY{n}{x}\PY{p}{,}\PY{n}{y}\PY{p}{,} \PY{n}{type}\PY{o}{=}\PY{l+s}{\PYZdq{}}\PY{l+s}{s\PYZdq{}}\PY{p}{)}
\end{Verbatim}
\end{tcolorbox}

    \begin{center}
    \adjustimage{max size={0.9\linewidth}{0.9\paperheight}}{output_162_0.png}
    \end{center}
    { \hspace*{\fill} \\}
    
    \begin{tcolorbox}[breakable, size=fbox, boxrule=1pt, pad at break*=1mm,colback=cellbackground, colframe=cellborder]
\prompt{In}{incolor}{94}{\boxspacing}
\begin{Verbatim}[commandchars=\\\{\}]
\PY{n+nf}{plot}\PY{p}{(}\PY{n}{x}\PY{p}{,}\PY{n}{y}\PY{p}{,} \PY{n}{type}\PY{o}{=}\PY{l+s}{\PYZdq{}}\PY{l+s}{s\PYZdq{}}\PY{p}{,} \PY{n}{main}\PY{o}{=}\PY{l+s}{\PYZdq{}}\PY{l+s}{Grafik\PYZdq{}}\PY{p}{)} \PY{c+c1}{\PYZsh{} nambah judul}
\end{Verbatim}
\end{tcolorbox}

    \begin{center}
    \adjustimage{max size={0.9\linewidth}{0.9\paperheight}}{output_163_0.png}
    \end{center}
    { \hspace*{\fill} \\}
    
    \begin{tcolorbox}[breakable, size=fbox, boxrule=1pt, pad at break*=1mm,colback=cellbackground, colframe=cellborder]
\prompt{In}{incolor}{95}{\boxspacing}
\begin{Verbatim}[commandchars=\\\{\}]
\PY{n+nf}{plot}\PY{p}{(}\PY{n}{x}\PY{p}{,}\PY{n}{y}\PY{p}{,} \PY{n}{type}\PY{o}{=}\PY{l+s}{\PYZdq{}}\PY{l+s}{s\PYZdq{}}\PY{p}{,} \PY{n}{main}\PY{o}{=}\PY{l+s}{\PYZdq{}}\PY{l+s}{grafik\PYZdq{}}\PY{p}{,} \PY{n}{xlab}\PY{o}{=}\PY{l+s}{\PYZdq{}}\PY{l+s}{sumbu x\PYZdq{}}\PY{p}{,} \PY{n}{ylab}\PY{o}{=}\PY{l+s}{\PYZdq{}}\PY{l+s}{sumbu y\PYZdq{}}\PY{p}{)}
\end{Verbatim}
\end{tcolorbox}

    \begin{center}
    \adjustimage{max size={0.9\linewidth}{0.9\paperheight}}{output_164_0.png}
    \end{center}
    { \hspace*{\fill} \\}
    
    \begin{tcolorbox}[breakable, size=fbox, boxrule=1pt, pad at break*=1mm,colback=cellbackground, colframe=cellborder]
\prompt{In}{incolor}{96}{\boxspacing}
\begin{Verbatim}[commandchars=\\\{\}]
\PY{n+nf}{plot}\PY{p}{(}\PY{n}{x}\PY{p}{,} \PY{n}{main}\PY{o}{=}\PY{l+s}{\PYZdq{}}\PY{l+s}{grafik\PYZdq{}}\PY{p}{,} \PY{n}{xlab}\PY{o}{=}\PY{l+s}{\PYZdq{}}\PY{l+s}{sumbu x\PYZdq{}}\PY{p}{,} \PY{n}{ylab}\PY{o}{=}\PY{l+s}{\PYZdq{}}\PY{l+s}{sumbu y\PYZdq{}}\PY{p}{)} \PY{c+c1}{\PYZsh{} nilai sumbu\PYZhy{}y bersifat opsional}
\PY{c+c1}{\PYZsh{} default x = y}
\end{Verbatim}
\end{tcolorbox}

    \begin{center}
    \adjustimage{max size={0.9\linewidth}{0.9\paperheight}}{output_165_0.png}
    \end{center}
    { \hspace*{\fill} \\}
    
    \begin{tcolorbox}[breakable, size=fbox, boxrule=1pt, pad at break*=1mm,colback=cellbackground, colframe=cellborder]
\prompt{In}{incolor}{97}{\boxspacing}
\begin{Verbatim}[commandchars=\\\{\}]
\PY{n+nf}{plot}\PY{p}{(}\PY{n}{df}\PY{p}{)} \PY{c+c1}{\PYZsh{} Plot seluruh dataset Iris}
\end{Verbatim}
\end{tcolorbox}

    \begin{center}
    \adjustimage{max size={0.9\linewidth}{0.9\paperheight}}{output_166_0.png}
    \end{center}
    { \hspace*{\fill} \\}
    
    \begin{tcolorbox}[breakable, size=fbox, boxrule=1pt, pad at break*=1mm,colback=cellbackground, colframe=cellborder]
\prompt{In}{incolor}{98}{\boxspacing}
\begin{Verbatim}[commandchars=\\\{\}]
\PY{c+c1}{\PYZsh{} Scatterplot individual}
\PY{n+nf}{plot}\PY{p}{(}\PY{n}{df}\PY{o}{\PYZdl{}}\PY{n}{Sepal.Length}\PY{p}{)}
\end{Verbatim}
\end{tcolorbox}

    \begin{center}
    \adjustimage{max size={0.9\linewidth}{0.9\paperheight}}{output_167_0.png}
    \end{center}
    { \hspace*{\fill} \\}
    
    \begin{tcolorbox}[breakable, size=fbox, boxrule=1pt, pad at break*=1mm,colback=cellbackground, colframe=cellborder]
\prompt{In}{incolor}{99}{\boxspacing}
\begin{Verbatim}[commandchars=\\\{\}]
\PY{n+nf}{plot}\PY{p}{(}\PY{n}{df}\PY{o}{\PYZdl{}}\PY{n}{Sepal.Width}\PY{p}{)}
\end{Verbatim}
\end{tcolorbox}

    \begin{center}
    \adjustimage{max size={0.9\linewidth}{0.9\paperheight}}{output_168_0.png}
    \end{center}
    { \hspace*{\fill} \\}
    
    Lebih lanjut, baca:
https://www.rdocumentation.org/packages/graphics/versions/3.6.2/topics/plot.

    \hypertarget{barplot}{%
\subsubsection{Barplot}\label{barplot}}

    \begin{tcolorbox}[breakable, size=fbox, boxrule=1pt, pad at break*=1mm,colback=cellbackground, colframe=cellborder]
\prompt{In}{incolor}{100}{\boxspacing}
\begin{Verbatim}[commandchars=\\\{\}]
\PY{n+nf}{barplot}\PY{p}{(}\PY{n}{df}\PY{o}{\PYZdl{}}\PY{n}{Sepal.Width}\PY{p}{)}
\end{Verbatim}
\end{tcolorbox}

    \begin{center}
    \adjustimage{max size={0.9\linewidth}{0.9\paperheight}}{output_171_0.png}
    \end{center}
    { \hspace*{\fill} \\}
    
    \begin{tcolorbox}[breakable, size=fbox, boxrule=1pt, pad at break*=1mm,colback=cellbackground, colframe=cellborder]
\prompt{In}{incolor}{101}{\boxspacing}
\begin{Verbatim}[commandchars=\\\{\}]
\PY{n+nf}{barplot}\PY{p}{(}\PY{n}{df}\PY{o}{\PYZdl{}}\PY{n}{Sepal.Width}\PY{p}{,} \PY{n}{col} \PY{o}{=} \PY{l+s}{\PYZsq{}}\PY{l+s}{red\PYZsq{}}\PY{p}{)}
\end{Verbatim}
\end{tcolorbox}

    \begin{center}
    \adjustimage{max size={0.9\linewidth}{0.9\paperheight}}{output_172_0.png}
    \end{center}
    { \hspace*{\fill} \\}
    
    \begin{tcolorbox}[breakable, size=fbox, boxrule=1pt, pad at break*=1mm,colback=cellbackground, colframe=cellborder]
\prompt{In}{incolor}{102}{\boxspacing}
\begin{Verbatim}[commandchars=\\\{\}]
\PY{c+c1}{\PYZsh{} jadi horizontal}
\PY{n+nf}{barplot}\PY{p}{(}\PY{n}{df}\PY{o}{\PYZdl{}}\PY{n}{Sepal.Width}\PY{p}{,} \PY{n}{col} \PY{o}{=} \PY{l+s}{\PYZsq{}}\PY{l+s}{red\PYZsq{}}\PY{p}{,} \PY{n}{horiz}\PY{o}{=}\PY{k+kc}{TRUE}\PY{p}{)}
\end{Verbatim}
\end{tcolorbox}

    \begin{center}
    \adjustimage{max size={0.9\linewidth}{0.9\paperheight}}{output_173_0.png}
    \end{center}
    { \hspace*{\fill} \\}
    
    Baca:
https://www.rdocumentation.org/packages/graphics/versions/3.6.2/topics/barplot.

    \hypertarget{boxplot}{%
\subsubsection{Boxplot}\label{boxplot}}

    Untuk lebih memahami apa itu boxplot, pembaca disarankan untuk
mengunjungi:
https://towardsdatascience.com/understanding-boxplots-5e2df7bcbd51.

    \begin{tcolorbox}[breakable, size=fbox, boxrule=1pt, pad at break*=1mm,colback=cellbackground, colframe=cellborder]
\prompt{In}{incolor}{103}{\boxspacing}
\begin{Verbatim}[commandchars=\\\{\}]
\PY{n+nf}{boxplot}\PY{p}{(}\PY{n}{df}\PY{o}{\PYZdl{}}\PY{n}{Sepal.Width}\PY{p}{,} \PY{n}{col} \PY{o}{=} \PY{l+s}{\PYZsq{}}\PY{l+s}{red\PYZsq{}}\PY{p}{)}
\end{Verbatim}
\end{tcolorbox}

    \begin{center}
    \adjustimage{max size={0.9\linewidth}{0.9\paperheight}}{output_177_0.png}
    \end{center}
    { \hspace*{\fill} \\}
    
    Baca:
https://www.rdocumentation.org/packages/graphics/versions/3.6.2/topics/boxplot.

    \hypertarget{histogram}{%
\subsubsection{Histogram}\label{histogram}}

    \begin{tcolorbox}[breakable, size=fbox, boxrule=1pt, pad at break*=1mm,colback=cellbackground, colframe=cellborder]
\prompt{In}{incolor}{104}{\boxspacing}
\begin{Verbatim}[commandchars=\\\{\}]
\PY{n+nf}{hist}\PY{p}{(}\PY{n}{df}\PY{o}{\PYZdl{}}\PY{n}{Sepal.Width}\PY{p}{,} \PY{n}{col}\PY{o}{=}\PY{l+s}{\PYZdq{}}\PY{l+s}{red\PYZdq{}}\PY{p}{)}
\end{Verbatim}
\end{tcolorbox}

    \begin{center}
    \adjustimage{max size={0.9\linewidth}{0.9\paperheight}}{output_180_0.png}
    \end{center}
    { \hspace*{\fill} \\}
    
    Baca:
https://www.rdocumentation.org/packages/graphics/versions/3.6.2/topics/hist.

    \hypertarget{sekilas-tentang-dasar-pembelajaran-mesin}{%
\section{Sekilas tentang dasar pembelajaran
mesin}\label{sekilas-tentang-dasar-pembelajaran-mesin}}

    \hypertarget{pengantar}{%
\subsection{Pengantar}\label{pengantar}}

    Pemrograman tradisional:

\begin{itemize}
\tightlist
\item
  Membaca nilai x\\
\item
  Membaca nilai y\\
\item
  \texttt{jum\ =\ x\ +\ y}
\item
  \texttt{print(jum)}
\end{itemize}

Jika pada pembelajaran mesin tidak demikian, kita masukan nilai - nilai
input dan output yang banyak, sehingga komputer dapat menemukan
algoritma pemecahan masalahnya sendiri tanpa harus diperintah.

Pemrograman tradisional: Data + Program \(\rightarrow\) Komputer
\(\rightarrow\) Output

Pembelajaran mesin: Data + Output \(\rightarrow\) Komputer
\(\rightarrow\) Program

\hypertarget{apa-yang-hendak-kita-lakukan-pada-sesi-ini}{%
\subsection{Apa yang hendak kita lakukan pada sesi
ini?}\label{apa-yang-hendak-kita-lakukan-pada-sesi-ini}}

Pada sesi ini kita akan membuat model yang dapat memprediksi lebar sepal
(\texttt{Sepal.Width}) dari dataset Iris berdasarkan panjang sepal
(\texttt{Sepal.Length}).

    \hypertarget{yuk-kita-mulai}{%
\subsection{Yuk kita mulai!}\label{yuk-kita-mulai}}

    \hypertarget{memuat-dataset}{%
\subsection{Memuat dataset}\label{memuat-dataset}}

    \begin{tcolorbox}[breakable, size=fbox, boxrule=1pt, pad at break*=1mm,colback=cellbackground, colframe=cellborder]
\prompt{In}{incolor}{105}{\boxspacing}
\begin{Verbatim}[commandchars=\\\{\}]
\PY{n+nf}{head}\PY{p}{(}\PY{n}{df}\PY{p}{)}
\end{Verbatim}
\end{tcolorbox}

    A data.frame: 6 × 5
\begin{tabular}{r|lllll}
  & Sepal.Length & Sepal.Width & Petal.Length & Petal.Width & Species\\
  & <dbl> & <dbl> & <dbl> & <dbl> & <fct>\\
\hline
	1 & 5.1 & 3.5 & 1.4 & 0.2 & Iris-setosa\\
	2 & 4.9 & 3.0 & 1.4 & 0.2 & Iris-setosa\\
	3 & 4.7 & 3.2 & 1.3 & 0.2 & Iris-setosa\\
	4 & 4.6 & 3.1 & 1.5 & 0.2 & Iris-setosa\\
	5 & 5.0 & 3.6 & 1.4 & 0.2 & Iris-setosa\\
	6 & 5.4 & 3.9 & 1.7 & 0.4 & Iris-setosa\\
\end{tabular}


    
    \hypertarget{memisahkan-data}{%
\subsubsection{Memisahkan data}\label{memisahkan-data}}

    \begin{tcolorbox}[breakable, size=fbox, boxrule=1pt, pad at break*=1mm,colback=cellbackground, colframe=cellborder]
\prompt{In}{incolor}{106}{\boxspacing}
\begin{Verbatim}[commandchars=\\\{\}]
\PY{n}{Sepalwidth} \PY{o}{=} \PY{n}{df}\PY{p}{[}\PY{p}{,}\PY{l+s}{\PYZdq{}}\PY{l+s}{Sepal.Width\PYZdq{}}\PY{p}{]}
\PY{n}{Sepallength} \PY{o}{=} \PY{n}{df}\PY{p}{[}\PY{p}{,}\PY{l+s}{\PYZdq{}}\PY{l+s}{Sepal.Length\PYZdq{}}\PY{p}{]}
\end{Verbatim}
\end{tcolorbox}

    \begin{tcolorbox}[breakable, size=fbox, boxrule=1pt, pad at break*=1mm,colback=cellbackground, colframe=cellborder]
\prompt{In}{incolor}{107}{\boxspacing}
\begin{Verbatim}[commandchars=\\\{\}]
\PY{n+nf}{print}\PY{p}{(}\PY{n}{Sepalwidth}\PY{p}{)}
\end{Verbatim}
\end{tcolorbox}

    \begin{Verbatim}[commandchars=\\\{\}]
  [1] 3.5 3.0 3.2 3.1 3.6 3.9 3.4 3.4 2.9 3.1 3.7 3.4 3.0 3.0 4.0 4.4 3.9 3.5
 [19] 3.8 3.8 3.4 3.7 3.6 3.3 3.4 3.0 3.4 3.5 3.4 3.2 3.1 3.4 4.1 4.2 3.1 3.2
 [37] 3.5 3.1 3.0 3.4 3.5 2.3 3.2 3.5 3.8 3.0 3.8 3.2 3.7 3.3 3.2 3.2 3.1 2.3
 [55] 2.8 2.8 3.3 2.4 2.9 2.7 2.0 3.0 2.2 2.9 2.9 3.1 3.0 2.7 2.2 2.5 3.2 2.8
 [73] 2.5 2.8 2.9 3.0 2.8 3.0 2.9 2.6 2.4 2.4 2.7 2.7 3.0 3.4 3.1 2.3 3.0 2.5
 [91] 2.6 3.0 2.6 2.3 2.7 3.0 2.9 2.9 2.5 2.8 3.3 2.7 3.0 2.9 3.0 3.0 2.5 2.9
[109] 2.5 3.6 3.2 2.7 3.0 2.5 2.8 3.2 3.0 3.8 2.6 2.2 3.2 2.8 2.8 2.7 3.3 3.2
[127] 2.8 3.0 2.8 3.0 2.8 3.8 2.8 2.8 2.6 3.0 3.4 3.1 3.0 3.1 3.1 3.1 2.7 3.2
[145] 3.3 3.0 2.5 3.0 3.4 3.0
    \end{Verbatim}

    \begin{tcolorbox}[breakable, size=fbox, boxrule=1pt, pad at break*=1mm,colback=cellbackground, colframe=cellborder]
\prompt{In}{incolor}{108}{\boxspacing}
\begin{Verbatim}[commandchars=\\\{\}]
\PY{n+nf}{print}\PY{p}{(}\PY{n}{Sepallength}\PY{p}{)}
\end{Verbatim}
\end{tcolorbox}

    \begin{Verbatim}[commandchars=\\\{\}]
  [1] 5.1 4.9 4.7 4.6 5.0 5.4 4.6 5.0 4.4 4.9 5.4 4.8 4.8 4.3 5.8 5.7 5.4 5.1
 [19] 5.7 5.1 5.4 5.1 4.6 5.1 4.8 5.0 5.0 5.2 5.2 4.7 4.8 5.4 5.2 5.5 4.9 5.0
 [37] 5.5 4.9 4.4 5.1 5.0 4.5 4.4 5.0 5.1 4.8 5.1 4.6 5.3 5.0 7.0 6.4 6.9 5.5
 [55] 6.5 5.7 6.3 4.9 6.6 5.2 5.0 5.9 6.0 6.1 5.6 6.7 5.6 5.8 6.2 5.6 5.9 6.1
 [73] 6.3 6.1 6.4 6.6 6.8 6.7 6.0 5.7 5.5 5.5 5.8 6.0 5.4 6.0 6.7 6.3 5.6 5.5
 [91] 5.5 6.1 5.8 5.0 5.6 5.7 5.7 6.2 5.1 5.7 6.3 5.8 7.1 6.3 6.5 7.6 4.9 7.3
[109] 6.7 7.2 6.5 6.4 6.8 5.7 5.8 6.4 6.5 7.7 7.7 6.0 6.9 5.6 7.7 6.3 6.7 7.2
[127] 6.2 6.1 6.4 7.2 7.4 7.9 6.4 6.3 6.1 7.7 6.3 6.4 6.0 6.9 6.7 6.9 5.8 6.8
[145] 6.7 6.7 6.3 6.5 6.2 5.9
    \end{Verbatim}

    \hypertarget{visualisasi-data}{%
\subsubsection{Visualisasi data}\label{visualisasi-data}}

    \begin{tcolorbox}[breakable, size=fbox, boxrule=1pt, pad at break*=1mm,colback=cellbackground, colframe=cellborder]
\prompt{In}{incolor}{109}{\boxspacing}
\begin{Verbatim}[commandchars=\\\{\}]
\PY{n+nf}{plot}\PY{p}{(}\PY{n}{Sepallength}\PY{p}{,} \PY{n}{Sepalwidth}\PY{p}{)}
\end{Verbatim}
\end{tcolorbox}

    \begin{center}
    \adjustimage{max size={0.9\linewidth}{0.9\paperheight}}{output_193_0.png}
    \end{center}
    { \hspace*{\fill} \\}
    
    \hypertarget{membangun-model-pembelajaran-mesin}{%
\subsubsection{Membangun model pembelajaran
mesin}\label{membangun-model-pembelajaran-mesin}}

    Karena hanya untuk pengenalan dasar, kita akan menggunakan algoritma
pembelajaran mesin paling sederhana, yakni regresi linier.

    \begin{longtable}[]{@{}ll@{}}
\toprule
x & y\tabularnewline
\midrule
\endhead
2 & 4\tabularnewline
4 & 8\tabularnewline
5 & 10\tabularnewline
6 & 12\tabularnewline
7 & 14\tabularnewline
8 & 16\tabularnewline
\bottomrule
\end{longtable}

    Bisa kah kalian lihat pola hubungan antara x dan y? \[y = 2x\]

Maka dengan demikian kita dapat menggunakan persamaan tersebut untuk
memprediksi nilai y untuk setiap x.

    Bentuk umum: \[y = a + bx\]

\[y = 2x \rightarrow A = 0, B =2\]

    Regresi linier digunakan untuk mencari nilai \(a\) dan \(b\) yang paling
mendekati persamaan analitik.

    \[Y = a + bX\],

di mana:

\[ a = \bar{y} - b\bar{x}\]

dan

\[b = \text{Cov}(x,y)/\text{Var}(x)\]

, di mana

\[\text{Cov}(x,y) = \frac{\Sigma(x_{i} - \bar{x})(y_{i} - \bar{y})}{n}\]

dan

\[\sigma^2 = \frac{\Sigma(x_{i}-\bar{x})}{N}\]

    Untuk penjelasan yang lebih mendetail tentang matematika di baliknya,
disarankan untuk melihat video Khan Academy berikut ini:

https://www.youtube.com/watch?v=ualmyZiPs9w

    \begin{tcolorbox}[breakable, size=fbox, boxrule=1pt, pad at break*=1mm,colback=cellbackground, colframe=cellborder]
\prompt{In}{incolor}{110}{\boxspacing}
\begin{Verbatim}[commandchars=\\\{\}]
\PY{c+c1}{\PYZsh{} regresi linier = lm(y \PYZti{} x)}
\PY{n}{model} \PY{o}{=} \PY{n+nf}{lm}\PY{p}{(}\PY{n}{Sepalwidth} \PY{o}{\PYZti{}} \PY{n}{Sepallength}\PY{p}{)}
\PY{n}{model}
\end{Verbatim}
\end{tcolorbox}

    
    \begin{verbatim}

Call:
lm(formula = Sepalwidth ~ Sepallength)

Coefficients:
(Intercept)  Sepallength  
    3.38864     -0.05727  

    \end{verbatim}

    
    \[y = 3.38864 -0.05727x\],

di mana:

\(y\) : \texttt{Sepalwidth}

\(x\) : \texttt{Sepallength}

    \hypertarget{memprediksi-berdasarkan-model-yang-telah-kita-bangun}{%
\subsubsection{Memprediksi berdasarkan model yang telah kita
bangun}\label{memprediksi-berdasarkan-model-yang-telah-kita-bangun}}

    \begin{tcolorbox}[breakable, size=fbox, boxrule=1pt, pad at break*=1mm,colback=cellbackground, colframe=cellborder]
\prompt{In}{incolor}{111}{\boxspacing}
\begin{Verbatim}[commandchars=\\\{\}]
\PY{n+nf}{predict}\PY{p}{(}\PY{n}{model}\PY{p}{,} \PY{n+nf}{data.frame}\PY{p}{(}\PY{n}{Sepallength}\PY{o}{=}\PY{l+m}{15}\PY{p}{)}\PY{p}{)}
\end{Verbatim}
\end{tcolorbox}

    \textbf{1:} 2.52961387253063

    
    Mustinya kita harus memisahkan data menjadi \emph{train - test sets},
tetapi karena webinar ini berfokus pada perkenalan R secara sederhana
maka hal tersbut tidak akan dibahas di sini.

    TERIMAKASIH


    % Add a bibliography block to the postdoc
    
    
    
\end{document}
